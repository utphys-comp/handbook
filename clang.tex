\chapter{C言語入門}

\noindent
プログラミング言語とは、コンピューターに仕事をさせるために、その作業内容を指示するための特別な言語である。この言語で書かれた作業の手順書を「プログラム」とよび、プログラムを作成することを「プログラミング」という。
プログラミング言語には、さまざまな種類がある。C言語はもともと、UNIX (マルチユーザ、マルチタスクのオペレーティングシステム)を記述するためのシステム言語として開発された。現在は、システムソフトウエアの作成だけでなく、事務処理や科学技術計算、アプリケーションソフトウエア(表計算やワープロなど)の開発など、広く汎用プログラミング言語として使用されている。

\section{C言語の基礎知識}
\label{sec:C:basic}
\subsection{なぜC言語?}

FortranやCくらいしかなかった時代と違い、最近では様々な(多くの場合Cよりもかなり書きやすい)プログラミング言語がある。そんな中どうしてわざわざ手間のかかるC言語を学ぶ必要があるのだろうか?

\clangpara{処理速度が速い}
多くの場合莫大な計算を行う必要のある科学技術計算において、処理速度が遅いというのは致命的である。Cは数あるプログラミング言語の中でも最速の部類に入り\footnote{太刀打ちできる言語はC\texttt{++}・Rust・Fortranくらいしかない。}基礎的な処理能力が非常に高いので、こういった用途に向く。

\clangpara{機能が必要最低限で把握しやすい}
Cの言語機能は非常に限定的であり(もちろんそれに起因する苦労もあるのだが)全体を比較的簡単に眺めることができる。エラーログも簡素で読みやすい。

\clangpara{様々な環境で利用できる}
PCだけではなく、ワークステーションやスーパーコンピュータ上でも利用可能である。環境依存性も少なく安心して利用できる。また、実験装置・解析装置でも使われるFPGAや組み込みコンピュータ環境においても利用できる。メモリの管理についてもユーザからよく見えるので、メモリ量が限られた環境での利用にも適している。

\clangpara{素直に実装できる}
低速な高級言語\footnote{Pythonなど。}では少々無理矢理にでもうまく計算が重い部分を外部ライブラリ(多くの場合Cで書かれている)に押し付ける工夫が必要になる。また、高速な高級言語\footnote{Juliaなど。}であってもバックエンドとして走るコンパイラにとって分かりやすい書き方をしなければ急速にパフォーマンスが落ちる。つまり、高効率なプログラムを書きたいときには``ただ考えた内容をそのままプログラムにする''だけではうまく行かないのである。対照的に、Cは基本的にあらゆる操作が充分高速であるのでこういった特殊な気遣いをする必要はあまりない。ただやりたいことをそのままプログラムに焼き直せば大抵の場合うまくいく。

\subsection{まずはコンパイル}

ここでは、最も短い``コンパイル可能なCのプログラム''を作成し、それをコンパイルしてみよう。

\clangpara{コンパイルとは?}
C, C\texttt{++}, Fortranなどの言語は「コンパイル言語」と呼ばれる。
これらの言語は、処理内容を記述したファイル(いわゆるソースコード)を直接実行することはせず、一度コンパイラというソフトウェアにソースコードを処理させ、出力された実行可能なファイルを実行対象とする。この操作をコンパイルという。

一方、コンパイルして計算機の言葉にするのではなく、ソフトウェアがプログラムを直接理解してその内容を実行するプログラム言語(スクリプト言語)もある。この「プログラムを直接理解するソフトウェア」のことをインタープリタといい、「プログラム」のことをスクリプトという。sh、Javascript、Ruby、Python、Juliaなどが代表的なインタープリタである。

性能が要求されるような状況下ではコンパイル言語を使用するのがよいが、事前にコンパイルしておかなくてはならない都合上あまり柔軟な操作はできない。
対してスクリプト言語は実行時まで曖昧さを残しておくことやインタラクティブな編集が出来る反面、動作速度に関してはコンパイル言語に及ぶべくもない。用途に応じて適切に使い分けよう。

\clangpara{型の概念について}
C言語の学習において最初の壁となるのは型の概念であろう。
高性能なプログラムを書く場合には変数の素性をある程度束縛する型の概念が不可欠となる。

主な変数の型としては以下のようなものがある。
\begin{table}[H]
    \centering
    \begin{tabular}{ll}
        \texttt{int}      & 整数型               \\
        \texttt{unsigned} & 符号なし整数型       \\
        \texttt{float}    & 単精度浮動小数点数型 \\
        \texttt{double}   & 倍精度浮動小数点数型 \\
        \texttt{char}     & 文字型
    \end{tabular}
\end{table} \noindent
``精度''は変数1つを記憶するのにどれだけのリソースを使うかに深く関係しており、一般にたくさんの bit を使う型のほうが精度や表現能力が高い。よって例えば\texttt{double} でもよい場合はわざわざ精度が低い \texttt{float} を使う利点はない。

各型の変数で表現できる値の範囲については状況依存であるが、典型的に以下のようになっている\footnote{整数型は1刻みで誤差がないのに対し、浮動小数点数型は値が大きくなるほど隣接する数字との間隔が増大して精度が低下することに注意せよ。}。
\begin{table}[H]
    \centering
    \begin{tabular}{ll}
        \texttt{int}      & \(-2^{31}\sim 2^{31}-1\)                                                                           \\
        \texttt{unsigned} & \(0\sim 2^{32}-1\)                                                                                 \\
        \texttt{float}    & {\small 0.0 と \(\pm(1.40129846432481707\times10^{-45}\sim 3.40282346638528860\times10^{+38}\))}   \\
        \texttt{double}   & {\small 0.0 と \(\pm(4.94065645841246544\times10^{-324}\sim 1.79769313486231570\times10^{+308}\))}
    \end{tabular}
\end{table} \noindent
\nendo 年現在、\texttt{int} は 32bit が主流となっている。

\clangpara{プログラムの構成}
Cのプログラムの最小単位は関数である。``Cのプログラムは関数を並べたものである''ということもできる。
関数を定義するときは以下のように書く。
\begin{quote}
    \begin{verbatim}
  型 関数名(引数) { 処理 }
  \end{verbatim}
\end{quote}
\begin{description}
    \item[関数名]\mbox{}\\
          関数の名前は自由につけられる。ここで、Cでプログラム中に使う名前(関数以外でも共通)には
          \begin{itemize}
              \item 先頭は英字
              \item 2文字以降は英数字または \texttt{\_} (下線)
              \item \texttt{\_} (下線)を2つ以上連続させてはならない
              \item 大文字/小文字は区別される
              \item 名前に重複があってはならない
          \end{itemize}
          という制約がある。
    \item[引数]\mbox{}\\
          関数には引数(パラメタ)を渡すことができるが、その型はあらかじめ指定しておかねばならない(具体的な例については\ref{sec:C:function}節参照)。
          渡さない場合には\texttt{void}と書く。
    \item[型]\mbox{}\\
          関数は何かを処理して最後に値を返す(関数値)もののことである。C言語の関数はあらかじめ決められた型の値しか返すことができない。
    \item[処理]\mbox{}\\
          この中で、1) 関数が行う処理内容を書き、2) 最後に関数値を返す。
          関数値を返すには
          \begin{quote}
\begin{verbatim}
return 関数値;
\end{verbatim}
          \end{quote}
          と書く。最終的にどれかに到達することさえ保証されていれば\texttt{return}が複数個あっても構わない。
\end{description}
プログラム中に関数は複数作ることができるが、必ず\texttt{main}という\texttt{int}を返す関数が必要である。この事情については後ほど説明することとして、ひとまずCにおける関数の例を示す。
\begin{reidai}\label{ex:hinagata1}
    \begin{verbatim}
int main(void) { return 0; }
\end{verbatim}
\end{reidai}

\clangpara{コンパイル}
Emacsなどのテキストエディタを使って例\ref{ex:hinagata1}の1行を入力したテキストファイルを作成し、\ref{ex:hinagata1}cという名前で保存する。このプログラムをコンパイルするにはC言語用のコンパイラ \texttt{gcc} を用いる。
\begin{commandline2}
    \prompt \underline{gcc \ref{ex:hinagata1}c -Wall -Wextra}
\end{commandline2} \noindent
\texttt{-Wall -Wextra}というのは警告メッセージを最大限出力させるためのオプションであり、必ずつけたほうがよい。
同一ディレクトリに\texttt{a.out}というファイルができているので、以下のように実行する\footnote{\texttt{.}がカレントディレクトリを表していたことを思い出そう。}。
\begin{commandline2}
    \prompt \underline{./a.out}
\end{commandline2} \noindent
(ただし、現段階のプログラムでは、実行しても何も起こらない。)

\clangpara{\texttt{main}について}
C言語において必ず必要になる\texttt{main}は``そのプログラムがどんな処理をするか''を記述した関数である。つまり、プログラムを実行した際に実行される部分は\texttt{main}の中身だけであり、他の部分に書いたものは\texttt{main}から何らかの形でアクセスしない限り完全に無視される。\texttt{main}が\texttt{int}型でなくてはならないのは歴史的経緯による\footnote{\texttt{main}が返した値は外部から終了コードとして参照できるようになっていて、\texttt{main}が正常に終了したかを確認することができる。正常終了したときには0を、異常終了したときにはそれ以外を返すのが一般的。}。

\subsection{書式の慣例}

先に進む前に、Cのプログラムの書式の慣例について、いくつか説明する。

\clangpara{自由書式 (free format)}
C言語では基本的に書式に制限がない(自由書式)。
文の終わりは、記号 \texttt{;} で判別する。
一つの行に二つの文を書くこともできるし、
\begin{quote}
\begin{verbatim}
aaaaa; bbbbbbb;
\end{verbatim}
\end{quote}
一つの文を二つの行にわけて書いても問題ない。
\begin{quote}
\begin{verbatim}
aaaaaaaa
aaaaaaa;
\end{verbatim}
\end{quote}
例えば、例\ref{ex:hinagata1}は、以下のようにも書くことができる。
\begin{reidai}\label{ex:hinagata2}
    \begin{verbatim}
int main(void)
{
    return 0;
}
\end{verbatim}
\end{reidai} \noindent
本書では、例\ref{ex:hinagata2}のような書式を用いることにする。

\clangpara{インデントの慣例}
Cの文は、意味上のレベル(入れ子構造の階層)を持つ。意味上の各レベルをわかりやすくするため、より深いレベルを右側に段付けして書くことが多い(インデント)。
\begin{quote}
    \begin{verbatim}
  aaaaa          /* ← 最上位のレベルの文 */
  aaaaa
    bbbbb        /* ← 一つ深いレベルの文 */
    bbbbb
      ccccc      /* ← もう一つ深いレベルの文 */
      ccccc
    bbbbb
  aaaaa
\end{verbatim}
\end{quote}
例\ref{ex:hinagata2}の関数は、すでにインデントして書かれている。
インデントは上の行から行毎に順にキーボード上の \tabkey を押せばできる。
\clangpara{コメント}
\texttt{/*} と \texttt{*/} で囲った部分はコメントとみなされる(複数行も可)。1行だけコメントアウトしたいときには\texttt{//}が便利。

例えば、例\ref{ex:hinagata2}にコメントを挿入してみると、
\begin{reidai}
    \begin{verbatim}
int main(void)
{
    /* これはコメント。
    複数行にわたっても OK。 */
    return 0;
}
\end{verbatim}
\end{reidai} \noindent
のようになり、\texttt{/*} と \texttt{*/} の間にある部分はコンパイルするときに無視される。

\clangpara{自動整形について}
プログラムの可読性を担保する上で、インデントやスペースの入れ方などの記法を統一することは極めて重要である。これを全て自分の手で行うのは非常に手間がかかるので、便利な自動整形ツールを使うとよい。C言語用としてはclang-formatが有名である\footnote{例えばPythonではautopep8やblack、\LaTeX ではlatexindentが使用できる。}。この冊子にあるプログラムはすべてclang-formatで整形されている。

\subsection{コンパイル(2)}

次に、``何かがおこる''プログラムをコンパイルしてみよう。
\begin{reidai}\label{ex:compile1}
    \begin{verbatim}
#include <stdio.h>

int main(void)
{
    printf("Hello, C!\n");
    return 0;
}
\end{verbatim}
\end{reidai} \noindent
例\ref{ex:compile1}のプログラムを Emacs などで作成し、\ref{ex:compile1}cという名前で保存する。ここで、\texttt{printf} は、文字を出力する標準ライブラリ関数である(詳細については後ほど説明する)。この例は``Hello, C!''を出力するもので、\texttt{\textbackslash n}は改行を表す特殊文字である。

それでは、保存したプログラムを、\texttt{gcc} を用いてコンパイルし、実行してみよう。
\begin{commandline2}
    \prompt \underline{gcc \ref{ex:compile1}c -Wall -Wextra} \\
    \prompt \underline{./a.out} \\
    Hello, C!
\end{commandline2} \noindent
コンパイル時にエラーで止まってしまう場合には、ソースコードをもう一度見直すこと。

また、次のようにすれば、\texttt{a.out} ではなく好きな名前の実行ファイルが作成できる。名前は\texttt{.out}で終わるものでなくてもよい。
\begin{commandline2}
    \prompt \underline{gcc -o hello \ref{ex:compile1}c -Wall -Wextra}
\end{commandline2} \noindent
この場合は同一ディレクトリに \texttt{hello} というファイルができ、以下のようにすれば実行できる。
\begin{commandline2}
    \prompt \underline{./hello} \\
    Hello, C!
\end{commandline2}

\clangpara{新しく出てきた概念について}
\begin{description}
    \item[\texttt{\#include <file>}]\mbox{}\\ \texttt{file}の中身をこの場所に展開する。ファイルを探索しに行く場所はあらかじめ決められており、そこに同名のファイルがなければエラーとなる。
    \item[\texttt{stdio.h}]\mbox{}\\ 標準ライブラリ関数\texttt{printf}関数を使う際に必要なファイル。
          システムの奥深くにある特別な場所に配置されており、特になんの対策も講じることなく使用できる。
\end{description}

\subsection{ライブラリのリンク}
プログラムの中で \(\sin x\) や \(\cos x\) などの数学関数を用いた場合は、前述の方法ではコンパイルに失敗する\footnote{数学関数の場合、macOSではエラーにならない。しかし、一般的なライブラリの中で定義されている関数を使う場合には適切なリンクが必要である。}。
まずは、例\ref{ex:compile2} のプログラムを作成し、\ref{ex:compile2}c という名前で保存しよう。
\begin{reidai}\label{ex:compile2}
    \begin{verbatim}
#include <math.h>
#include <stdio.h>

int main(void)
{
    const double angle = 60.0 * M_PI / 180.0;
    printf("cos(%lf) is %lf\n", angle, cos(angle));
    return 0;
}
\end{verbatim}
\end{reidai} \noindent
このプログラムでは\texttt{math.h} 内で宣言されている数学関数を使っているため、\texttt{libm} というライブラリが必要となる。
ライブラリとは何かについてはさておくこととして、以下のようにすることでコンパイルができる。
\begin{commandline2}
    \prompt \underline{gcc \ref{ex:compile2}c -lm -Wall -Wextra}
\end{commandline2} \noindent
\texttt{-lm}という部分は、\texttt{m} (libm のうち lib を削除した残り)を \texttt{-l} というオプション引数を使って \texttt{gcc} に渡す構文である。例えば、libsocket を使いたい場合は、\texttt{-lsocket} と指定する。\texttt{math.h} の中には sin や cos だけでなく、例\ref{ex:compile2}で使われている \(\pi\) (=3.1415...)も \texttt{M\_PI}として定義されている。
定数\texttt{M\_PI}はCの標準機能ではないため、\texttt{gcc}以外のコンパイラを使用したり、\texttt{gcc}にオプションをつけて使用するとエラーとなることがあるので注意せよ。

\clangpara{ライブラリについて}
ライブラリとは、大雑把に言うと``既にコンパイルがほぼ済んだ状態で準備された関数''を呼び出すために必要なファイルである。
今回使用したlibmは数学関数をまとめて定義したもので、\texttt{math.h}とはまた別の特別な場所に格納されており、\texttt{-l} オプションでリンクするだけで呼び出すことができるようになる。

\clangpara{新しく出てきた概念について}
\begin{description}
    \item[\texttt{math.h}] \mbox{}\\
          数学関数をまとめて定義したヘッダ。使用するには\texttt{libm}が必要。
    \item[算術演算子] \mbox{} \\
          以下の5種類がある。
          \begin{table}[H]
              \centering
              \begin{tabular}{ll}
                  加算 & \texttt{+}  \\
                  減算 & \texttt{-}  \\
                  乗算 & \texttt{*}  \\
                  除算 & \texttt{/}  \\
                  剰余 & \texttt{\%}
              \end{tabular}
          \end{table} \noindent
          整数を整数で割るとあまりを切り捨てる整数の割り算になることと\footnote{例えば、\texttt{3.0/2}は\text{1.5}となるが、\texttt{3/2}は\texttt{1}になる。}、冪乗演算子は存在しないことに注意せよ。
    \item[変数定義] \mbox{}\\
          \texttt{const double angle = 60.0 * M\_PI / 180.0;}の部分で\texttt{angle}という変数を定義して代入している。今回は先頭に\texttt{const}をつけたのでこれ以降この変数を変更することはできない\footnote{変更する予定のない変数には積極的に\texttt{const}をつけるようにしよう。これだけでかなりデバッグがしやすくなる。}。
\end{description}
\clangpara{\texttt{printf}について}
例\ref{ex:compile2}だけでなく例\ref{ex:compile1}でも出てきた、\texttt{printf} という関数(\textbf{print} with \textbf{f}ormattingの略)は画面に文字や数字を表示させる場合に用いられ、
\begin{quote}
    \begin{verbatim}
printf("フォーマット", 変数1, 変数2,......)
\end{verbatim}
\end{quote}
という形で使用する。例\ref{ex:compile2}の場合はフォーマットが ``\texttt{cos(\%lf) is \%lf\textbackslash n}'' になっているので、前者の \texttt{\%lf} の部分が変数1 (\texttt{angle})の値に置きかえられ、後者の \texttt{\%lf} の部分が変数2 (\texttt{cos(angle)})の値に置きかえられて表示される。\texttt{\%lf} の他に \texttt{\%d}, \texttt{\%s}, \texttt{\%f} などがあり、変数の型によって使い分ける必要がある。
\begin{table}[H]
    \centering
    \begin{tabular}{ll}
        \texttt{\%d}  & \texttt{int}          \\
        \texttt{\%u}  & \texttt{unsigned}     \\
        \texttt{\%f}  & \texttt{float}        \\
        \texttt{\%lf} & \texttt{double}       \\
        \texttt{\%c}  & \texttt{char} 文字    \\
        \texttt{\%s}  & \texttt{char*} 文字列
    \end{tabular}
\end{table}
\texttt{\%}の部分では型だけでなく書式の指定もできる。例えば\texttt{\%.20lf}とすると小数点以下20桁が表示されるようになる。

ここから先では他にも様々な標準ライブラリ関数を使用するが、その詳細については基本的に省略する。\texttt{man}やWeb検索で調べてほしい。

\section{制御文}
\subsection{Cにおける論理値}
古い規格のCには真(\texttt{true})と偽(\texttt{false})の2値しかとらない論理型(いわゆる\texttt{bool})がない\footnote{新しい規格では一応存在することになっているが、言語機能として備わっているとは言い難い。}。そこで真の代わりに\(\neq 0\) (典型的に\(1\)を用いるが、それ以外でも構わない)、偽の代わりに\(0\)を用いる。

以下の比較演算子・関係演算子が使用できる。
\begin{table}[H]
    \centering
    \begin{tabular}{ll}
        \texttt{a} と \texttt{b} が等しい     & \texttt{a == b} \\
        \texttt{a} と \texttt{b} は等しくない & \texttt{a != b} \\
        \texttt{a} は \texttt{b} より大きい   & \texttt{a > b}  \\
        \texttt{a} は \texttt{b} より小さい   & \texttt{a < b}  \\
        \texttt{a} は \texttt{b} 以上         & \texttt{a >= b} \\
        \texttt{a} は \texttt{b} 以下         & \texttt{a <= b}
    \end{tabular}
\end{table} \noindent
浮動小数点に対して\texttt{==}を用いると誤差無しの完全一致を要求すること(数値誤差があるのでほとんど確実に偽である)に注意せよ。

また、論理演算子には以下のようなものがある。
\begin{table}[H]
    \centering
    \begin{tabular}{ll}
        \texttt{a} または \texttt{b} & \texttt{a || b}   \\
        \texttt{a} かつ \texttt{b}   & \texttt{a \&\& b} \\
        \texttt{a} でない            & \texttt{!a}       \\
    \end{tabular}
\end{table}

\subsection{ブロックについて}
\texttt{\{\}}で囲われた部分をブロックという\footnote{既に関数定義の書式として出てきた。}。新しい規格のC言語では変数の宣言をあらゆる場所に書くことができる。あるブロックの中で定義された変数はそのブロックの中でしか使えないが、逆に外にある変数をブロックの中で参照したり、変更することはできる。
この性質をうまく用いることで、プログラム中のごく限られた領域でしか使用しないはずの変数を誤って他の場所で使ってしまうという厄介なバグをかなり抑制することができる。

\subsection{\texttt{if}文}
「...ならば...を実行して、それ以外なら...を実行する」
という内容のプログラムは \texttt{if} 〜 \texttt{else} 〜を用いて書く。
\begin{reidai}\label{ex:if}
    \begin{verbatim}
#include <stdio.h>

int main(void)
{
    const int a = 10;
    const int b = 20;
    if (a == b)
    {
        printf("a is equal to b\n");
    }
    else
    {
        printf("a is not equal to b\n");
    }
    return 0;
}
\end{verbatim}
\end{reidai} \noindent
この例では、\texttt{a} と \texttt{b} が異なるので、実行すると\texttt{a is not equal to b}と表示される。

\texttt{if} 文は一般に以下のような形をとる。
\begin{quote}
\begin{verbatim}
if (A)
{
    ブロック A
}
else if (B)
{
    ブロック B
}
...
else
{
    ブロック C
}
\end{verbatim}
\end{quote} \noindent
処理が\texttt{if}に到達すると上から順番に\texttt{()}の中身がチェックされていき、一番最初に条件を満たしたブロックの中を実行してそれ以降は無視する(\texttt{else}まで来たらブロック Cを実行する)。\texttt{else if}はいくつあってもよいし、\texttt{else}はなくても構わない。

単純な\texttt{if}をより簡潔に書くための構文として三項演算子(\texttt{?:})というものがある。調べてみよ。

\subsection{\texttt{for}文}
繰り返しを行うためには、\texttt{for}を用いる。
次の例は1から100までの和を計算するものである。
\begin{reidai}\label{ex:for}
\begin{verbatim}
#include <stdio.h>

int main(void)
{
    int sum = 0;
    for (int i = 1; i <= 100; ++i)
    {
        sum += i;
    }
    printf("sum of integers from 1 to 100 is %d\n", sum);
    return 0;
}
\end{verbatim}
\end{reidai} \noindent
\texttt{for}文は以下のような形になる。
\begin{quote}
\begin{verbatim}
for (A; B; C)
{
    ブロック
}
\end{verbatim}
\end{quote}
処理が\texttt{for}に到達するとまず\texttt{A}が実行され、それ以降は``\texttt{B}を評価\(\rightarrow\) \texttt{B}が真ならブロックの中身を実行し、偽なら\texttt{for}から抜ける\(\rightarrow\) \texttt{C}を実行''の組み合わせを繰り返す。\texttt{A,B,C}は空欄でも構わない。

例\ref{ex:for}における\texttt{int i}のような、\texttt{for}ループのカウンターとして使用する変数は\texttt{A}の部分で定義するのが好ましい\footnote{ここで定義した変数は\texttt{for}のブロック中でのみ有効なので、他の場所で誤って参照したり名前が衝突する心配がなくなる。}。

\clangpara{新しく出てきた概念について}
\begin{description}
    \item[\texttt{++}]\mbox{}\\
          \texttt{++i} は \texttt{i = i + 1} と同じ意味を持っている。この \texttt{++} をインクリメント演算子という。同様に、\texttt{i = i - 1} と同じ意味を持つものとして \texttt{{-}{-}i} がある。これをデクリメント演算子という。この他にも、\texttt{i++} と \texttt{i{-}{-}} という書き方もある。これらの違いは、インクリメントされたりデクリメントされたりするタイミングが異なることにある。もし、
          \begin{quote}
              \begin{verbatim}
int i = 10;
int j = i++;
\end{verbatim}
          \end{quote}
          ならば、\texttt{j}の値は11ではなく10となる(\texttt{j}に代入した後で\texttt{i}を1増やす)。ところが、
          \begin{quote}
              \begin{verbatim}
int i = 10;
int j = ++i;
\end{verbatim}
          \end{quote}
          ならば、\texttt{j} の値は10ではなく11となる(\texttt{i}を1増やしてから\texttt{j}に代入する)。
          特別な理由がない限り、\texttt{++i} と \texttt{{-}{-}i} を使うのがよい\footnote{\texttt{i++}は一度もとの\texttt{i}の値を別の場所に保存しておく必要があるので少しだけ計算量が多い。}。
    \item[複合代入]\mbox{}\\
          今回使用した\texttt{a += b}は\texttt{a = a + b}と同じ意味の式である。\texttt{+=}を使ったほうが短く書ける上、場合によっては前者のほうが高速になる場合もある\footnote{前者は\texttt{a}を直接置き換えてよいからである。Cでは残念ながら大きな影響はない。}ので、可能な限りこちらを使うほうがよい。

          全く同様の記法が加減乗除と剰余全てについて使用できる。
\end{description}


\subsection{\texttt{while}文}
ある条件が満たされている間、何かを繰り返し実行したいときは、\texttt{while}を使う。次の例は、\texttt{i}が0より大きい間は\texttt{while}ブロックを実行し、その結果として、1から100までの和を求めるものである\footnote{今回は\texttt{while}を用いたが、本来は\texttt{for}で実装するべき操作である。}。
\begin{reidai}\label{ex:while}
\begin{verbatim}
#include <stdio.h>

int main(void)
{
    int i = 100;
    int sum = 0;
    while (i > 0)
    {
        sum += i;
        --i;
    }
    printf("sum of integers from 100 to 1 is %d\n", sum);
    return 0;
}
\end{verbatim}
\end{reidai} \noindent
\texttt{while} 文は次のように動作する。
\begin{quote}
\begin{verbatim}
while (A)
{
    ブロック
}
\end{verbatim}
\end{quote}
\texttt{while}に処理が到達すると、``\texttt{A}を評価\(\rightarrow\) \texttt{A}が真ならブロックを実行''の組み合わせを繰り返す(初回チェックで\texttt{A}が偽ならブロックの中身は一度も実行されない)。
\texttt{while(1)}と書くとAが常に正しいとして扱われるので、次に説明する\texttt{break}を呼ぶまで何度でもループし続ける。

\subsection{\texttt{break}文}
\texttt{for}や\texttt{while}のような繰り返しを行う文の中から途中で抜けたい場合には、\texttt{break}を用いる。例\ref{ex:break}では、\(\cos\)の値が負になったとき強制的に\texttt{for}文を抜けている。
\begin{reidai}\label{ex:break}
    \begin{verbatim}
#include <math.h>
#include <stdio.h>

int main(void)
{
    for (int i = 0; i <= 180; i += 20)
    {
        const double angle = i * M_PI / 180.0;
        const double cosval = cos(angle);
        printf("cos(%lf) is %lf\n", angle, cosval);
        if (cosval < 0.0)
        {
            break;
        }
    }
    return 0;
}
\end{verbatim}
\end{reidai}

\subsection{\texttt{continue} 文}
ある条件を満たした場合に\texttt{for}や\texttt{while}の先頭に戻り、次の繰り返しに進みたい場合には、\texttt{continue}を用いる(より正確には、\texttt{continue}から後を実行せずに、次の条件判定を行う)。 例\ref{ex:continue}では、割り切れない場合には\texttt{continue}を実行してすぐに \texttt{i <= number} の判定を行っている。割り切れる場合は \texttt{printf(...)} を実行してから \texttt{i <= number} の判定を行っている。ちなみに、このプログラムは80の全ての約数を出力するものである。
\begin{reidai}\label{ex:continue}
    \begin{verbatim}
#include <stdio.h>

int main(void)
{
    const int number = 80;
    for (int i = 1; i <= number; ++i)
    {
        /* 'a % b' yields the residual of a/b */
        const int residual = number % i;
        if (residual != 0)
        {
            continue;
        }
        printf("%d is a measure(YAKUSUU) of %d\n", i, number);
    }
    return 0;
}
\end{verbatim}
\end{reidai}
\begin{renshuu}\label{prob:2-1}
    \texttt{if}文の練習として、\(ax+b=0\) の解を求めるプログラムを書け。
    \(a\), \(b\), \(x\) を表示させて終了させること。また、\(a\), \(b\) はあらかじめプログラムの中で決めてよい。
\end{renshuu}

\begin{renshuu}\label{prob:2-2}
    \texttt{for} あるいは \texttt{while} の練習として、\(n!\) (階乗)を求めるプログラムを書け。
    \(n\)と結果を表示させて終了させること。また、\(n\)はあらかじめプログラムの中で決めてよい。
\end{renshuu}

\begin{renshuu}\label{prob:2-3}
    2から100までの素数を表示するプログラムを書け。
\end{renshuu}

\section{配列}
\subsection{1次元配列}
1次元配列を用いるときには、
\begin{quote}
    \begin{verbatim}
int a[10];
\end{verbatim}
\end{quote}
のように宣言する。この場合、
\begin{quote}
    \begin{verbatim}
a[0], a[1],...,a[9]
\end{verbatim}
\end{quote}
の10個の要素を持つ配列が生成される。添字が0から始まっていること、定義した時点では初期化がされていない(何らかの値がすでに入っていることがある)ことに注意せよ。例\ref{ex:array1d}では、\texttt{\#define}によって\texttt{N\_ELEMENT}を10と定義している\footnote{詳しくは、参考書などの「プリプロセッサ」に関する説明を参照のこと。\texttt{N\_ELEMENT}はプログラムの実行中に変化する数ではなく、コンパイル時に定まっている定数である。}。\texttt{for}文によって、配列\texttt{array}に数字が格納されている。あとはそれを順番に表示して終了する。なお、
\begin{quote}
    \begin{verbatim}
int n = 10;
int a[n];
\end{verbatim}
\end{quote}
はコンパイルエラーとなる。この記法を使うときは、配列のサイズを動的(プログラム実行時)に決めることはできない\footnote{1999年改訂のC標準規格C99では認められているが、2011年改訂のC11では必須機能でなくなってしまった。}。入力値などに応じてサイズを動的に決めたい場合は、\texttt{malloc}と\texttt{free} (\ref{sec:clang:malloc}節)を用いる。

なお、大きすぎる配列(概ね 8MB 以上が目安)を確保するとクラッシュすることがあるので注意せよ。
\begin{reidai}\label{ex:array1d}
    \begin{verbatim}
#include <stdio.h>
#define N_ELEMENT 10

int main(void)
{
    int array[N_ELEMENT];
    for (int i = 0; i < N_ELEMENT; ++i)
    {
        array[i] = i * i * i;
    }
    for (int i = 0; i < N_ELEMENT; ++i)
    {
        printf("array[%d] = %d\n", i, array[i]);
    }
    return 0;
}
\end{verbatim}
\end{reidai}

\subsection{2次元配列}
2次元配列は、
\begin{quote}
    \begin{verbatim}
int a[2][3];
\end{verbatim}
\end{quote}
のように宣言する。この場合、
\begin{quote}
    \begin{verbatim}
a[0][0], a[0][1], a[0][2]
a[1][0], a[1][1], a[1][2]
\end{verbatim}
\end{quote}
の六つの要素を持つ2次元配列が生成される。例\ref{ex:array2d}は、\(2 \times 2\)の行列の逆行列を求めるものである。
\begin{reidai}\label{ex:array2d}
    \begin{verbatim}
#include <stdio.h>

int main(void)
{
    const double matrix[2][2]
        = { { 10.0, 5.0 },
              { 8.0, 14.0 } };
    /* matrix =
    | 10.0 5.0 |
    | 8.0 14.0 | */
    const double det = matrix[0][0] * matrix[1][1]
        - matrix[0][1] * matrix[1][0];
    if (det == 0.0)
    {
        printf("inverse matrix does not exist\n");
    }
    else
    {
        const double inverse[2][2]
            = { { matrix[1][1] / det, -matrix[0][1] / det },
                  { -matrix[1][0] / det, matrix[0][0] / det } };
        printf("inverse[0][0] = %lf\n", inverse[0][0]);
        printf("inverse[0][1] = %lf\n", inverse[0][1]);
        printf("inverse[1][0] = %lf\n", inverse[1][0]);
        printf("inverse[1][1] = %lf\n", inverse[1][1]);
    }
    return 0;
}
\end{verbatim}
\end{reidai}
初期化は括弧を用いることでまとめて行うことができる。

\begin{renshuu}\label{prob:3-1}
    \texttt{int}型で大きさが5の1次元配列 \texttt{a} と \texttt{b} を準備し、配列 \texttt{a} にあらかじめ数字を代入しておく。その配列 \texttt{a} の要素をすべて配列 \texttt{b} に代入し、その後、配列 \texttt{a} をすべて 0 にするプログラムを書きなさい。\texttt{a} と \texttt{b} をすべて表示させて終了すること。
\end{renshuu}

\begin{renshuu}\label{prob:3-2}
    \texttt{double} 型で大きさが \(2\times2\) の2次元配列 \texttt{a} と \texttt{b} を準備し、その積を計算するプログラムを書きなさい。\texttt{a} と \texttt{b} とその積を表示させて終了すること。だたし、 \texttt{a} と \texttt{b} はあらかじめプログラムの中で決めてよい。
\end{renshuu}

\section{文字列と標準入力}
文字列の取り扱いは非常に複雑である。ここではごく基本的なことに限定して解説する。また、標準入力(キーボードなどからの入力)を受け付け、それをプログラムで利用する方法についても説明する。

\subsection{文字列}
文字列とは、名前通り文字の配列である。文字列を扱うためには、\texttt{char s[10];} のように宣言する。ただし、次のような形の代入はできない。
\begin{quote}
    \begin{verbatim}
char s[10];
s = "Hello";
\end{verbatim}
\end{quote}
文字列の宣言と同時に文字を代入するには次のように行う。
\begin{quote}
    \begin{verbatim}
char s[10] = "Hello";
\end{verbatim}
\end{quote}
あるいは、
\begin{quote}
    \begin{verbatim}
char s[] = "Hello";
\end{verbatim}
\end{quote}
後者は自動的に \texttt{[]} の中は 6 になる。(前者の 10 の部分は適当な数字でよいが、\texttt{Hello} 5文字を記憶するためには 6 以上の数字を用いる必要がある。) では、なぜ \texttt{[]} の中は6になるのか? なぜ5でないのか? という疑問が生じるだろう。これには以下のような事情が関係している。

文字列を使う場合、どこでその文字列が終了するかを判断できなければならない。ところが、ここで使っている \texttt{s} という変数は単にその文字列の先頭を示しているだけで、
どこで終わるかという情報を持っていない(\ref{sec:C:pointer-array}節参照)。そのため、C言語では文字列の最後に必ず ``\texttt{\textbackslash 0}''という特殊な文字をつけるという決まりがある。そうすることで、文字列の最後を判断することが可能になるわけだ。このような理由で、5ではなく6になるのである。
\begin{figure}[H]
    \centering
    \resizebox{0.80\textwidth}{!}{\includegraphics{hello.pdf}}
\end{figure}

\subsection{標準入力(1) : \texttt{gets}関数}
例\ref{ex:gets}を作成しコンパイルしてみよう(コンパイラによっては、「\texttt{gets} は危険だ」と警告が出るかもしれないが、実行ファイルはちゃんとできているはずである)。これを実行すると、\texttt{Input:} と表示され入力を促される。ここで、\texttt{uni}などと入力すると、\texttt{uni}と表示されてプログラムが終了する。この場合、\texttt{gets}という関数によって入力された文字列が \texttt{str} にコピーされる。注意すべきことは、\texttt{str} が \texttt{str[20]} と宣言されているので入力すべき文字列は19文字以内に限られるということである。しかし実際には、ユーザーは20字以上入力することができる。その場合、はみ出た文字は用意された領域の外に書き込まれる。もしそこが、別の変数用に使われていたら? これはとても危険なことである。\texttt{gets}を使う場合は注意が必要である\footnote{どうしても\texttt{gets}が使用したいときにはより安全性が高い\texttt{fgets}を使用せよ。}。
\begin{reidai}\label{ex:gets}
    \begin{verbatim}
#include <stdio.h>

int main(void)
{
    char str[20];
    printf("Input: ");
    gets(str);
    printf("%s\n", str);
    return 0;
}
\end{verbatim}
\end{reidai}

\subsection{標準入力(2) : \texttt{scanf}関数}
例\ref{ex:gets}ではたとえ数字(19桁以下)を入力してもそれは文字列として扱われるため、数字として足し算などに用いるためには \texttt{atoi}などの関数を用いる必要がある。別の方法として、\texttt{scanf} を用いて入力した数字をそのまま数字として扱う方法を例\ref{ex:scanf}に示す(ソースコード中の \texttt{\&i} や \texttt{\&x} についている \texttt{\&} については後述する)。
\begin{reidai}\label{ex:scanf}
    \begin{verbatim}
#include <stdio.h>
#include <string.h>

int main(void)
{
    int i;
    printf("Input(int): ");
    scanf("%d", &i);
    printf("%d\n", i);
    double x;
    printf("Input(double): ");
    scanf("%lf", &x);
    printf("%lf\n", x);
    return 0;
}
\end{verbatim}
\end{reidai}

\begin{renshuu}\label{prob:4-1}
    練習\ref{prob:2-2}で、標準入力から \(n\) を決定し、その答えを表示するプログラムを書きなさい。
\end{renshuu}

\section{ポインタ}

ポインタの習得はC言語の習得の中で一つの大きな壁である。はじめは訳がわからないかもしれないが、C言語の基本的な部分の一つなので、いろいろな例に触れて慣れてほしい。慣れが一番である。

\subsection{とりあえず(1)}

例\ref{ex:pointer1}を作成し実行してみよう。上手くいけば、``\texttt{q is 200 and *p is 200.}''と表示されるはずである。この例で出てくる \texttt{p} がポインタである。\texttt{int} 型のポインタを宣言するには、
\begin{quote}
    \begin{verbatim}
int *p;
\end{verbatim}
\end{quote}
のように\texttt{*}をつける。人によっては、
\begin{quote}
    \begin{verbatim}
int* p;
\end{verbatim}
\end{quote}
と宣言したほうがイメージしやすいかもしれない。つまり、\texttt{p} は \texttt{int*} 型(\texttt{int} のポインタ)ということである(しかし、2個以上のポインタを宣言するためには、\texttt{int *p, *q;} としなければならない。\texttt{int* p, q;} とすると\texttt{q}は\texttt{int}型になってしまう)。

例\ref{ex:pointer1}の \texttt{q} に格納されている\(200\)はコンピュータのメモリーのどこかに電気的に存在するはずである。その格納場所を示すものがアドレス(言うなればメモリー上の住所)であり、これを\texttt{\&q}で表す。今回はこれを\texttt{p}へと代入したので、\texttt{p}は\texttt{q}が存在する場所を指し示すようになる。

変数からそのアドレスを抽出する\texttt{\&}と対照的なのが\texttt{*}であり、これは逆にポインタからその指している先そのものを抽出する\footnote{\texttt{*}は型から1つアスタリスクを剥がし、逆に\texttt{\&}は1つ増やす、と覚えるとよい。}。\texttt{p}はあくまで\texttt{q}を指し示しているだけ(\texttt{q}が存在するアドレスの情報しか持っていない)なので、実体\texttt{q}を利用する際には\texttt{*p}とせねばならないのである。

\begin{reidai}\label{ex:pointer1}
    \begin{verbatim}
#include <stdio.h>

int main(void)
{
    /* qを初期化 */
    int q = 200;
    /* qのアドレスをpに代入する */
    int* p = &q;
    /* pから指している対象そのもの(q)を取り出す */
    printf("q is %d and *p is %d.\n", q, *p);
    return 0;
}
\end{verbatim}
\end{reidai}
\begin{figure}[H]
    \centering
    \resizebox{0.50\textwidth}{!}{\includegraphics{pointer.pdf}}
\end{figure}

\subsection{とりあえず(2)}
では、もう一つ例を示そう。例\ref{ex:pointer2}では、\texttt{p = \&q;} によりポインタ代入を行ったので \texttt{p} というポインタで \texttt{q} にアクセスできる。ここでは、\texttt{q} に値を代入するかわりに、\texttt{*p}を用いて値を代入している。\texttt{p}はポインタで、\texttt{*p} は \texttt{p} が指している対象である\texttt{q}そのものである。
実行結果は、``\texttt{q is 300 and *p is 300.}''となる。
\begin{reidai}\label{ex:pointer2}
    \begin{verbatim}
#include <stdio.h>

int main(void)
{
    int q;
    int* p = &q;
    /* 値を取得するだけでなく書き換えることもできる */
    *p = 300;
    printf("q is %d and *p is %d.\n", q, *p);
    return 0;
}
\end{verbatim}
\end{reidai} \noindent
記号\texttt{*}, \texttt{\&}などの意味と役割をもう一度復習しておこう。
\begin{table}[H]
    \centering
    \begin{tabular}{l}
        \texttt{int *p;} \\
        \texttt{int q;}
    \end{tabular}
\end{table} \noindent
の時、
\begin{table}[H]
    \centering
    \begin{tabular}{ll}
        \texttt{p}   & ポインタ (つまりどこかのアドレス) \\
        \texttt{*p}  & 実体                              \\
        \texttt{q}   & 実体                              \\
        \texttt{\&q} & \texttt{q} のアドレス
    \end{tabular}
\end{table} \noindent
である。

注意すべきことは、例\ref{ex:pointer1}や例\ref{ex:pointer2}で \texttt{p = \&q;} がない場合は \texttt{p} が\textbf{どこを指しているか未定義}であるということだ。したがって、事前の初期化なしに\texttt{*p = 300;} のように書き換えを試みたり、 \texttt{printf} 内で \texttt{*p} を表示させるのは危険である。何が起こるか分からない。

この節では、静的に宣言された変数を指すためにポインタを用いてきたが、より高度な使い方については\ref{sec:clang:malloc}節で紹介する。

\subsection{ポインタと配列}
\label{sec:C:pointer-array}
ポインタと配列には密接な関係がある。
\begin{quote}
    \begin{verbatim}
int array[10];
\end{verbatim}
\end{quote}
と宣言した場合、\texttt{array[0]} や \texttt{array[5]} などで配列の要素にアクセスすることができた。実は、\texttt{array} だけでも意味を持つ。\texttt{array} は配列型 (\texttt{int[10]} 型) の変数で、プログラムコード中に登場した時は、大抵の状況では、自動的に ``配列の先頭要素のアドレス''に暗黙的に変換されて解釈される。つまり、\texttt{array} は \texttt{array[0]} へのポインタとして解釈され、\texttt{*array} は \texttt{array[0]}を意味する。さらに、ポインタに整数を足すとその数だけ先の要素を指すようになる。例えば\texttt{array + 2}は\texttt{array[2]} へのポインタであり、\texttt{*(array + 2)} は \texttt{array[2]}と等価である。

では、例\ref{ex:pointer-array}を見てみよう。\texttt{strlen} という関数は \texttt{string.h} 内で宣言されている文字列の長さを返す関数である。最初の \texttt{for} 文は文字型の配列 \texttt{str} の中身を順番に書き出している。次の \texttt{for} 文がポイントである。ここではまず文字型のポインタ \texttt{p} に文字型の配列 \texttt{str} の先頭ポインタ \texttt{str} を代入している。そして、文字型の実体である \texttt{*p} が \texttt{\textbackslash 0} でない場合はその \texttt{*p} を表示し、ポインタ \texttt{p} の指す部分を一つ進めている。これを \texttt{*p} が \texttt{\textbackslash 0} になるまで繰り返すことで、最初の \texttt{for} 文と同じ結果を表示できるというわけである。
\begin{reidai}\label{ex:pointer-array}
    \begin{verbatim}
#include <stdio.h>
#include <string.h>

int main(void)
{
    const char str[] = "ABCDE";
    const int num = strlen(str);
    for (int i = 0; i < num; ++i)
    {
        printf("%c", str[i]);
    }
    printf("\n");
    for (const char* p = str; *p != '\0'; ++p)
    {
        printf("%c", *p);
    }
    printf("\n");
    return 0;
}

\end{verbatim}
\end{reidai}

\subsection{\texttt{const}とポインタについて}
ポインタを介して変数を操作出来るのは便利な反面、意図せず書き換えてしまった結果バグを生じたり、そもそも\texttt{const}のついた変数をどうやってポインタで参照するのか(うっかり書き換えてしまったりしないのか)といった問題が生じる。これを解決するのが\texttt{const}つきのポインタである。

\begin{table}[H]
    \centering
    \begin{tabular}{lll}
        型                         & ポインタ自体を変更できるか & そのポインタから実体を変更できるか \\
        \texttt{type*}             & \(\bigcirc\)               & \(\bigcirc\)                       \\
        \texttt{const type*}       & \(\bigcirc\)               & \(\times\)                         \\
        \texttt{type* const}       & \(\times\)                 & \(\bigcirc\)                       \\
        \texttt{const type* const} & \(\times\)                 & \(\times\)
    \end{tabular}
\end{table} \noindent

少し難しい概念ではあるが、バグを防ぐ上で非常に強力であるので困ったときにはぜひ使ってみてほしい。
\begin{renshuu}\label{prob:5-1}
    次のプログラムの \texttt{printf} の部分を配列ではなく、ポインタを使ったものに書き換えなさい(\texttt{array} を使っても、あるいは、\texttt{int *p;} を加えてもよい)。
\begin{quote}
\begin{verbatim}
#include <stdio.h>

int main(void)
{
    int array[10];
    for (int i = 0; i < 10; ++i)
    {
        array[i] = i * i;
    }
    for (int i = 0; i < 10; ++i)
    {
        printf("array[%d] = %d\n", i, array[i]);
    }
    return 0;
}
\end{verbatim}
\end{quote}
\end{renshuu}

\section{複素数}
\label{sec:C:complex}
量子力学などの数値計算では複素数を用いることも多い。
Cにおいても複素数変数を扱うための型が用意されている\footnote{複素数の機能はC11で必須でなくなってしまった。}。
\begin{table}[H]
    \centering
    \begin{tabular}{ll}
        \texttt{float complex}  & 単精度複素数型 \\
        \texttt{double complex} & 倍精度複素数型
    \end{tabular}
\end{table} \noindent
複素数型を使うためには、プログラムの冒頭で\texttt{complex.h}ヘッダファイルを読み込む必要がある。\texttt{csin}, \texttt{cexp}, \texttt{clog}などの複素初等関数も用意されている\footnote{単精度複素数の場合は、\texttt{csinf}, \texttt{cexpf}, \texttt{clogf}のように関数名の後に \texttt{f}を付ける。}。以下の例では、\(\exp(i\pi)\)の計算を行っている。
\begin{reidai}\label{ex:complex}
    \begin{verbatim}
#include <complex.h>
#include <math.h>
#include <stdio.h>

int main(void)
{
    const double complex x = 0 + 1 * I; /* 虚数単位 */
    const double complex y = cexp(x * M_PI);
    printf("i = (%lf,%lf)\n", creal(x), cimag(x));
    printf("e^{i*pi} = (%lf,%lf)\n", creal(y), cimag(y));
    return 0;
}
\end{verbatim}
\end{reidai} \noindent
虚数単位は大文字の \texttt{I} と書く。あるいは、大文字の\texttt{CMPLX}関数\footnote{単精度複素数の場合、\texttt{CMPLXF}。}を使って、\texttt{x = CMPLX(0,1)} のように表すこともできる。複素数の実部、虚部を取り出すにはそれぞれ、\texttt{creal}, \texttt{cimag}を使う。
絶対値、偏角、複素共役はそれぞれ、\texttt{cabs}, \texttt{carg}, \texttt{conj}である\footnote{単精度複素数の場合、それぞれ \texttt{crealf}, \texttt{cimagf}, \texttt{cabsf}, \texttt{cargf}, \texttt{conjf}である。}。

\clangpara{\texttt{tgmath.h}について}
C言語では関数名に重複があってはならないので、数学関数が型ごとに違う名前を持っている。これをいちいち切り替えるのは面倒であるが、新しい規格においては勝手に引数の型を読み取って適切な関数を呼び出してくれる便利な\texttt{tgmath.h}というものが追加されている。

\section{関数}
\label{sec:C:function}

\ref{sec:C:basic}節でC言語の基本構造を説明したが、\texttt{main(){...}} だけでプログラムを書くことはほとんどない。大きなプログラムになればなるほど関数化を行い、役割分担をはっきりさせるのがよい。

\subsection{あまり良くない例}
例\ref{ex:non-function}は円の面積の計算するために直接その公式を書いている。しかし、何度も何度も円の面積を計算したいとき、半径を入力すれば面積が返ってくるような関数があれば非常に便利である。
\begin{reidai}\label{ex:non-function}
    \begin{verbatim}
#include <math.h>
#include <stdio.h>

int main(void)
{
    const double radius = 2.0;
    const double area = radius * radius * M_PI;
    printf("Radius: %lf, Area: %lf\n", radius, area);
    return 0;
}
\end{verbatim}
\end{reidai}

\subsection{関数化}
では、例\ref{ex:non-function}を例\ref{ex:function1}のように変更してみよう。このような小さなプログラムでは関数化の効力は乏しいが、大きくなればなるほど、また複雑になればなるほど、その効力は絶大になる。例\ref{ex:function1}では、\texttt{circle\_area} という関数が定義されている。この関数では、double型の値を引数として受け取り、面積を計算してその値を戻り値にしている。

\begin{reidai}\label{ex:function1}
    \begin{verbatim}
#include <math.h>
#include <stdio.h>

double circle_area(double r)
{
    return r * r * M_PI;
}

int main(void)
{
    const double radius = 2.0;
    const double area = circle_area(radius);
    printf("Radius: %lf, Area: %lf\n", radius, area);
    return 0;
}
\end{verbatim}
\end{reidai}

通常、関数は使う前に``宣言''する必要がある。(例\ref{ex:function1}は特殊な場合であり、関数の定義が宣言を兼ねている)
きちんと宣言を行うときには、例\ref{ex:function2}のように関数の引数名と処理部分がない不完全な定義のようなもの(これが宣言の書式である)を書く必要がある。

関数を使用する際には使用箇所から宣言が見えるようにしておく必要があるが、逆に宣言さえ見えていれば定義はどこにあっても構わない\footnote{より厳密に言うならば、\texttt{gcc}に渡す\texttt{.c}ファイルのどれか、あるいはリンクするライブラリに定義が含まれていれば問題ない。}。
例\ref{ex:function2}は、``宣言は見えるが、定義は見えない''場合の典型例になっている。

\begin{reidai}\label{ex:function2}
    \begin{verbatim}
#include <math.h>
#include <stdio.h>

double circle_area(double);

int main(void)
{
    const double radius = 2.0;
    const double area = circle_area(radius);
    printf("Radius: %lf, Area: %lf\n", radius, area);
    return 0;
}

double circle_area(double r)
{
    return r * r * M_PI;
}
\end{verbatim}
\end{reidai}

\subsection{ポインタを引数にする関数}

例\ref{ex:function1}では関数の返り値が面積の一つだけであった。しかし、例えば割り算で商と余りを返したい場合、例\ref{ex:function1}のような関数ではうまく実現できない。なぜなら、戻り値は1個しか指定できないからである。これを解決するには、ポインタを引数とする関数を作り、あらかじめ準備しておいた答えを入れる変数を関数の引数にポインタとして渡し、関数内でそこに答えを入れてもらって関数の外で受け取るという方法が有効である。重要な点は、ポインタでなければ関数内で代入した値を関数の外で使うことはできないということである。

例\ref{ex:function-pointer}では、まず\texttt{main}で答えを入れてもらうための \texttt{shou} と \texttt{amari} という箱(変数)を準備している。次に、そのポインタ(アドレス)を関数\texttt{division}に渡している。そして、\texttt{division}内で答えを代入してもらい、関数の外で\texttt{printf}を用いて答えを表示している。\texttt{division} という関数の先頭にある\texttt{void}という型は(返り値の形では)何も返さないことを示すためのものである。
\begin{reidai}\label{ex:function-pointer}
    \begin{verbatim}
#include <stdio.h>

void division(int divident, int divisor,
    int* const quotient, int* const residual)
{
    *quotient = divident / divisor;
    *residual = divident % divisor;
}

int main(void)
{
    const int josuu = 3;
    const int hi_josuu = 13;
    int shou, amari;
    division(hi_josuu, josuu, &shou, &amari);
    printf("%d / %d = %d ... %d\n", hi_josuu, josuu, shou, amari);
    return 0;
}
\end{verbatim}
\end{reidai} \noindent
ポインタを使う理由がはっきりと分からない場合は、例\ref{ex:function-non-pointer}を作って実行してみてほしい。間違った答えが表示されるだろう。なぜなら、この場合 \texttt{division} という関数では、\texttt{quotient}と\texttt{residual}という二つの一時変数が生成され、それらに計算結果が代入された直後に破棄されるからである。一時変数は関数の処理が終了すると同時に破棄され、値は \texttt{main} の中の \texttt{shou} と \texttt{amari} には引き継がれない\footnote{ブロック中での変数定義の挙動を思い出そう。}。このため、関数に外部から値を渡すだけであれば、例\ref{ex:function-non-pointer}のような引数の宣言方法でよい(「値渡し」と呼ぶ)が、内での計算結果を関数の外で使いたい場合は、ポインタを使って変数のアドレスを渡す(「ポインタ渡し」と呼ぶ)必要がある。
\begin{reidai}\label{ex:function-non-pointer}
    \begin{verbatim}
#include <stdio.h>

void division(int divident, int divisor, int quotient, int residual)
{
    quotient = divident / divisor;
    residual = divident % divisor;
}

int main(void)
{
    const int josuu = 3;
    const int hi_josuu = 13;
    int shou, amari;
    division(hi_josuu, josuu, shou, amari);
    printf("%d / %d = %d ... %d\n", hi_josuu, josuu, shou, amari);
    return 0;
}
\end{verbatim}
\end{reidai}

\subsection{関数ポインタ}
ポインタはメモリ上にある変数を指し示すものであったが、実は関数そのものもメモリ上に配置されているのでポインタが使える。例えば\texttt{int}を受け取って\texttt{double}を返す関数用のポインタ\texttt{p}を宣言したいときには\texttt{double (*p)(int);}と書けばよい\footnote{変数宣言らしからぬ特殊な記法であるが、これであっている。}。

関数ポインタを使うと``関数を引数として受け取る関数''\footnote{いわゆる汎関数のようなものを作ることができる。}が実現でき、うまく使うと非常に強力である。

\subsection {Fortranの関数や手続きの利用}

LAPACK\footnote{行列の対角化、連立1次方程式の求解など線形計算を行うライブラリ。ほぼ全ての計算機で利用可能である。}などの多くのライブラリがFortranで書かれている。これらと同等の機能を持つ関数をCやC\texttt{++}で作ることもできるが、既に存在するのであればそちらを使った方が便利である。ここでは、C言語の中からFortranの関数や手続きを呼ぶ方法を簡単に紹介する。

Fortranの関数や手続きの名前をC言語から呼ぶためには、その名前をすべて小文字にして、最後に\_ (下線)を付けなければならない。また、関数などの引数の型も適切に読み変える必要がある。例えば、
\begin{reidai}\label{ex:fort}
    \begin{verbatim}
      Real*8 CALC(I, X, Y)
      Interger*4 I
      Real*4 X
      Real*8 Y
\end{verbatim}
\end{reidai} \noindent
というFortranの関数が存在するとする。このときCでは以下のような宣言を書く必要がある。
\begin{reidai}
    \begin{verbatim}
extern double calc_(int*, float*, double*);

int main(void)
{
    int i = 10;
    float x = 20.0F;
    double y = 30.0;
    double ret = calc_(&i, &x, &y);
    return 0;
}
\end{verbatim}
\end{reidai} \noindent
全ての引数をポインタ渡しとしなければならないことに特に注意せよ。コンパイルはC (\texttt{gcc})とFortran (\texttt{gfortran})で別々に行い、最後にリンクすればよい\footnote{LAPACKはすでにコンパイル済みなので\texttt{gfortran}は必要ない。リンクの方法については\ref{ex:ext_lib}を見よ。}。

\section{構造体}
データ構造を取り扱うためには構造体を用いる。

例\ref{ex:struct1}を見てほしい。個人のデータを取り扱うために \texttt{personal\_data} という一つの箱を準備している。これが構造体である。もし構造体を使わなければ、同じ内容のプログラムを実現するために複数の配列を準備する必要が出てくる。これは見た目にも格好悪いし、拡張性も乏しく複雑になる。例\ref{ex:struct1}では、\texttt{struct personal\_data pdata;} で \texttt{pdata} を構造体とし宣言している。少し分かりにくいかもしれないが、\texttt{int n;}と比べてみると、\texttt{int} と ``\texttt{struct personal\_data}''が同じ立場であって、\texttt{int} と ``\texttt{struct}''が同じ立場というわけではないことに注意してほしい。\texttt{int} という型はあらかじめC言語で決められた型なのに対し、構造体は自分が好きなように使いやすいように決めた型だと考えても構わない。つまり、``\texttt{struct personal\_data}''型というものを自分で作ったと考えるのである。そうすれば、\texttt{struct personal\_data pdata;} という使い方も理解できるはずである。

構造体内の各データには、ドット演算子 (\texttt{.}) を使ってアクセスする。また、年齢を文字列 \texttt{buffer[16]} として受け取っているため、\texttt{atoi} 関数を用いて数字に変換している。\texttt{atoi} は文字列を整数に変換する関数である。詳しくはコマンドラインで \texttt{man atoi} としてみよ。
\begin{reidai}\label{ex:struct1}
    \begin{verbatim}
#include <stdio.h>
#include <stdlib.h>

struct personal_data
{
    char family_name[16];
    char given_name[16];
    int age;
};

int main(void)
{
    struct personal_data pdata;
    char buffer[16];
    printf("Input family name: ");
    gets(pdata.family_name);
    printf("Input given name: ");
    gets(pdata.given_name);
    printf("Input age: ");
    gets(buffer);
    pdata.age = atoi(buffer);
    printf("Family Name = %s\n", pdata.family_name);
    printf("Given  Name = %s\n", pdata.given_name);
    printf("Age         = %d\n", pdata.age);
    return 0;
}
\end{verbatim}
\end{reidai} \noindent
構造体を指すポインタの場合は、次の例\ref{ex:struct2}のように、アロー演算子 (\texttt{->}) でその構造体の要素にアクセスすることができる。例題中の\texttt{qdata->p[0]}は\texttt{(*qdata).p[0]}と等価である。ちなみに、この例の構造体は粒子の質量と運動量を格納している。
\begin{reidai}\label{ex:struct2}
    \begin{verbatim}
#include <stdio.h>

struct particle_data
{
    double mass;
    double p[3]; /* Momentum */
};

int main(void)
{
    const struct particle_data pdata = { 0.14, { 1.2, 1.3, 1.4 } };
    const struct particle_data* const qdata = &pdata;
    printf("Mass = %lf\n", qdata->mass);
    printf("Momentum = (%lf, %lf, %lf)\n",
        qdata->p[0], qdata->p[1], qdata->p[2]);
    return 0;
}
\end{verbatim}
\end{reidai}

\begin{renshuu}\label{prob:7-1}
    ``月'', ``日'', ``1月1日からの日数(うるう年ではない)''を構成要素とする構造体を作って、``月'' と ``日''を標準入力から入力すれば、自動的に ``1月1日からの日数''の部分を埋め、最後にそれを表示して終了するプログラムを書きなさい。
\end{renshuu}

\section{ファイルの取り扱い}

ファイルから数字を読みこんだり、ファイルに結果を書き込んだりするプログラムを紹介する。ここで挙げた例だけでも基本的なことはできるはずである。より詳しく知りたい場合は参考書などを参照してほしい。

\subsection{ファイルから数字の読み込み}

例\ref{ex:file-read1}がファイルから数字を読み込むときの雛型となる。まず、\texttt{FILE *fp;} でファイルを取り扱うための構造体を宣言する。\texttt{char *filename}の部分でファイル名を宣言する。そして、\texttt{fopen}でファイルを開く。(\texttt{"r"} はread (読み取り専用)でファイルを開くことを示す。) \texttt{if(fp==NULL){}} の部分はエラーが起こったときに処理され、プログラムを強制終了させる\footnote{\texttt{NULL}は特別なポインタで、``なにか異常が起きた結果、ポインタが有効な場所を指さずに宙ぶらりんになっている''ことを表現するためにある。今回の例ではファイルが何らかの原因で読み取れなかったときに\texttt{NULL}が返ってくる。}。\texttt{fscanf}で順番にファイルに書かれている数字を\texttt{double}型で読み取っている。\texttt{fscanf}の使い方は、
\begin{quote}
    \begin{verbatim}
fscanf(ファイル構造体のポインタ, "フォーマット", &変数1, &変数2, ......);
\end{verbatim}
\end{quote}
で、ファイル構造体のポインタ部分以外は \texttt{scanf} と同じである。最後に \texttt{fclose(fp);} でファイルを閉じている。
\begin{reidai}\label{ex:file-read1}
    \begin{verbatim}
#include <stdio.h>
#include <stdlib.h>

double product(double x1, double y1, double x2, double y2)
{
    return x1 * y1 + x2 * y2;
}

int main(void)
{
    const char filename[] = "vectors.txt";
    FILE* fp = fopen(filename, "r");
    if (fp == NULL)
    {
        printf("Can't open file %s\n", filename);
        exit(1);
    }
    double x1, y1, x2, y2;
    /* read first line */
    fscanf(fp, "%lf %lf\n", &x1, &y1);
    /* read second line */
    fscanf(fp, "%lf %lf\n", &x2, &y2);
    fclose(fp);
    const double p = product(x1, y1, x2, y2);
    printf("Product of (%lf,%lf) and (%lf,%lf) is %lf\n", x1, y1, x2, y2, p);
    return 0;
}
\end{verbatim}
\end{reidai} \noindent
例\ref{ex:file-read1}を実行するためには、\texttt{vectors.txt}というファイル名で以下のような内容の入力ファイルも作っておく必要がある。
\begin{quote}
    \begin{verbatim}
 1.0    2.0
-1.5    3.5
\end{verbatim}
\end{quote}
例\ref{ex:file-read1}のデータを読み込む部分を、例\ref{ex:file-read2}のように \texttt{while} を使って書き直してみよう。こちらの方がファイルに存在するデータを最後まで読み取るという点で、より汎用性が高い。\texttt{fscanf}はデータの読み込みに失敗すると\texttt{EOF}を返すので、\texttt{EOF}が返ってくるまで読み込みを繰り返している\footnote{\texttt{while}の条件式チェックと読み込みを同時に行っている。}。
\begin{reidai}\label{ex:file-read2}
    \begin{verbatim}
#include <stdio.h>
#include <stdlib.h>

#define N_DATA 100

int main(void)
{
    const char filename[] = "vectors.txt";
    FILE* fp = fopen(filename, "r");
    if (fp == NULL)
    {
        printf("Can't open file %s\n", filename);
        exit(1);
    }
    double x[N_DATA][2];
    int index = 0;
    /* read data */
    while (fscanf(fp, "%lf %lf\n", &x[index][0], &x[index][1]) != EOF)
    {
        printf("Data %d : (%lf, %lf)\n", index, x[index][0], x[index][1]);
        ++index;
    }
    fclose(fp);
    return 0;
}
\end{verbatim}
\end{reidai}

\subsection{ファイルへの数字の書き出し}

例\ref{ex:file-write}がファイルへ数字を書き出すときの雛型である。ファイルを開くところまでは例\ref{ex:file-read1}とはほぼ同じで、違う点は \texttt{fopen} の第二引数が \texttt{"r"} ではなく \texttt{"w"} (書きこみ)である点である。書き出すときには \texttt{fprintf} という関数を使っている。\texttt{fprintf} の使い方は、
\begin{quote}
    \begin{verbatim}
fprintf(ファイル構造体のポインタ, "フォーマット", 変数1, 変数2, ......);
\end{verbatim}
\end{quote}
で、ファイル構造体のポインタ部分以外は \texttt{printf} と同じである。最後に、\texttt{fclose(fp);} でファイルを閉じる。
\begin{reidai}\label{ex:file-write}
    \begin{verbatim}
#include <math.h>
#include <stdio.h>
#include <stdlib.h>

double circle_area(double r)
{
    return r * r * M_PI;
}

int main(void)
{
    const char filename[] = "circle_area.txt";
    FILE* fp = fopen(filename, "w");
    if (fp == NULL)
    {
        printf("Can't open file %s\n", filename);
        exit(1);
    }
    for (int i = 1; i <= 10; ++i)
    {
        const double radius = i;
        const double area = circle_area(radius);
        fprintf(fp, "%lf -> %lf\n", radius, area);
    }
    fclose(fp);
    return 0;
}
\end{verbatim}
\end{reidai}

\section{その他の制御文}

\texttt{for}, \texttt{while}以外の制御文の書式を紹介する。

\subsection{\texttt{switch}-\texttt{case}文}

1の場合にはAを、2の場合にはBを、3の場合にはCを、という風に条件分岐をしたい場合に、\texttt{switch}-\texttt{case}文を用いる。
\begin{reidai}\label{ex:case}
    \begin{verbatim}
#include <stdio.h>
#include <string.h>

int main(void)
{
    int i;
    printf("Input(int): ");
    scanf("%d", &i);
    printf("%d / 3 no amari ha ", i);
    switch (i % 3)
    {
    case 1:
        printf("1 desu.\n");
        break;
    case 2:
        printf("2 desu.\n");
        break;
    default:
        printf("0 desu.\n");
    }
    return 0;
}
\end{verbatim}
\end{reidai} \noindent
今回は\texttt{i \% 3}をチェックの対象とし、上から\texttt{1}のとき、\texttt{2}のとき、それ以外の順で処理を記述している。もし例\ref{ex:case}の \texttt{switch}-\texttt{case} 文中に \texttt{break;} がないと、\texttt{case 1:} の場合は \texttt{case 2:} の部分も \texttt{default:} の部分も実行してしまうことに注意\footnote{\texttt{-Wextra}をつけていればきちんと警告が出るので安心してよい。}。(\texttt{case 1:} 以下の部分から下の部分をすべて実行する。)したがって、\texttt{case 2:} の直前で抜けるためには \texttt{break;} が必要となる。実際に \texttt{break;} を削除して試してみよう。この場合は常に \texttt{...0 desu.} が表示されるはずである。

\texttt{switch}-\texttt{case} 文を簡単にまとめると、次のようになる。
\begin{quote}
    \begin{verbatim}
switch (X) {
  case A:  ブロック1
  case B:  ブロック2
  ...
  default: ブロックN
}
\end{verbatim}
\end{quote}
\texttt{switch}文に処理が到達すると、\texttt{X}が上から順に\texttt{case}と照合されていき、一致したところでブロックに入る。\texttt{switch}から抜けるには\texttt{break;}を使用する。\texttt{case}はいくつあっても構わない。

\subsection{\texttt{do}-\texttt{while}文}

\texttt{while} 文と同様に、ある条件が満たされている場合に繰り返しを行うために \texttt{do}-\texttt{while} 文を使う。\texttt{while} 文と違う点は、条件の判定する``タイミング''である。\texttt{while} 文の場合は、まず条件を判定し、満足していれば\texttt{while} 内を実行するが、\texttt{do}-\texttt{while} 文の場合は、まず\texttt{do}-\texttt{while} 内を実行してから条件を判定する。したがって、\texttt{while} 文の場合は \texttt{while} 内を1度も実行しない場合があるが、\texttt{do}-\texttt{while} 文の場合は必ず1度は \texttt{do}-\texttt{while} 内を実行する。
\begin{reidai}\label{ex:do-while}
    \begin{verbatim}
#include <stdio.h>

int main(void)
{
    int i = 100;
    int sum = 0;
    do
    {
        sum += i;
        --i;
    } while (i != 0);
    printf("sum of integers from 100 to 1 is %d\n", sum);
    return 0;
}
\end{verbatim}
\end{reidai} \noindent
例\ref{ex:do-while}のように、\texttt{while()} の後ろの \texttt{;} を忘れないこと。
\texttt{do}-\texttt{while} 文の基本的な動作は次のようになる。
\begin{quote}
    \begin{verbatim}
do
{
  ブロック
} while (A);
\end{verbatim}
\end{quote}
\texttt{do}に処理が到達すると、``ブロックを実行\(\rightarrow\)\texttt{A}をチェックし、偽なら抜ける''の操作が繰り返される。

\section{コマンドライン引数の受け渡し}
\texttt{main}文には、\texttt{int main(void)} 以外に、\texttt{int main(int, char**)}, あるいは \texttt{int main(int, char*[])} という書き方がある。これらの書式を利用すれば、実行時に引数を与えてそれを利用することができる。つまり、今までは
\begin{commandline2}
    \prompt \underline{./a.out}
\end{commandline2} \noindent
として実行していたが、これを利用すれば、
\begin{commandline2}
    \prompt \underline{./a.out input.data output.data}
\end{commandline2} \noindent
のように、ファイル名や数値などをコマンドライン引数としてプログラムに渡すことができる。

\subsection{ポインタ配列}

新しい\texttt{main}関数の説明の前に、引数の受け渡しに使われているポインタ配列に関して説明しておこう。ポインタ配列は、名前通り、ポインタを要素とする配列(\texttt{char*[]})である。配列はポインタを使ってアクセス可能なので、``ポインタのポインタ'' (\texttt{char**})と同様に使うことができる。
\begin{reidai}\label{ex:pointers-array}
    \begin{verbatim}
#include <stdio.h>

int main(void)
{
    char* name0 = "Alice";
    char* name1 = "Bob";
    char* name2 = "Claire";
    char* name3 = "David";
    char* name[4];
    name[0] = name0;
    name[1] = name1;
    name[2] = name2;
    name[3] = name3; /* A */
    for (int i = 0; i < 4; ++i)
    { /* B */
        printf("Name%d : %s\n", i, name[i]);
    }
    for (int i = 0; i < 4; ++i)
    { /* C */
        printf("Name%d : %c, %s\n", i, **(name + i), *(name + i));
    }
    return 0;
}
\end{verbatim}
\end{reidai} \noindent
例\ref{ex:pointers-array}の結果は以下のようになる。
\begin{quote}
    \begin{verbatim}
Name0 : Alice
Name1 : Bob
Name2 : Claire
Name3 : David
Name0 : A, Alice
Name1 : B, Bob
Name2 : C, Claire
Name3 : D, David
\end{verbatim}
\end{quote}
まず、\texttt{name[0]} の型は \texttt{char*} なので、\texttt{name0} の値などを代入することができる。Aの時点で、\texttt{name} というポインタ配列の各要素にアドレスの代入したことになる。つまり、
\begin{quote}
    \begin{verbatim}
name[0] = "Alice"という文字列の先頭アドレス
name[1] = "Bob"という文字列の先頭アドレス
name[2] = "Claire"という文字列の先頭アドレス
name[3] = "David"という文字列の先頭アドレス
\end{verbatim}
\end{quote}
となっている。したがって、Bの \texttt{for} 文では、\texttt{printf} に文字列の先頭アドレスを渡すことで、名前を表示することができる。次に、Cの \texttt{for} 文である。これは若干混乱を招くが、次が理解出来れば、何とかクリアできるだろう。
\begin{table}[H]
    \centering
    \begin{tabular}{r@{ = }c@{ = }l}
        \texttt{name}   & 文字へのポインタのポインタ & \texttt{char**} 型                      \\
        \texttt{*name}  & 文字へのポインタ           & \texttt{char*} 型                       \\
        \texttt{**name} & 文字                       & \texttt{char} 型 (※ 文字列ではなく文字)
    \end{tabular}
\end{table}
\texttt{printf} は \texttt{\%s} で文字列の先頭アドレス、つまり、\texttt{char*} を受け取り、\texttt{\%c} で文字型そのもの、つまり、\texttt{char} を受け取る。また、\texttt{name + i} という書き方は\texttt{\&(name[i])}とほぼ等価である。

\subsection{新しい\texttt{main}関数}

例\ref{ex:new-main}がコマンドライン引数を受け取ることのできる \texttt{main}関数の例である。\texttt{int main(int argc, char* argv[])}は、\texttt{int main(int argc, char** argv)}と書いても同じである。好きなほうを使えばよい。
また、伝統的に \texttt{argc} と \texttt{argv} という名前を使っているが、他の変数名でも構わない。
\begin{reidai}\label{ex:new-main}
    \begin{verbatim}
#include <stdio.h>
#include <stdlib.h>

int main(int argc, char* argv[])
{
    if (argc != 3)
    { /* A */
        printf("Usage: sumXY InputFile OutputFile\n");
        exit(1);
    }
    const char* const InputFileName = argv[1];
    const char* const OutputFileName = argv[2];
    /* input */
    FILE* fp = fopen(InputFileName, "r");
    if (fp == NULL)
    {
        printf("Can't open file %s\n", InputFileName);
        exit(1);
    }
    int counter = 0;
    double x, y;
    double sumX = 0.0, sumY = 0.0;
    /* read data */
    while (fscanf(fp, "%lf %lf\n", &x, &y) != EOF)
    {
        sumX += x;
        sumY += y;
        ++counter;
    }
    fclose(fp);
    /* output */
    fp = fopen(OutputFileName, "w");
    if (fp == NULL)
    {
        printf("Can't open file %s\n", OutputFileName);
        exit(1);
    }
    /* write results */
    fprintf(fp, "Number of Data = %d\n", counter);
    fprintf(fp, "X Sum = %lf\n", sumX);
    fprintf(fp, "Y Sum = %lf\n", sumY);
    fclose(fp);
    return 0;
}
\end{verbatim}
\end{reidai} \noindent
例\ref{ex:new-main}を\ref{ex:new-main}cという名前で保存し、次のようにコンパイルする。
\begin{commandline2}
    \prompt \underline{gcc -o sumXY \ref{ex:new-main}c -Wall -Wextra}
\end{commandline2} \noindent
\texttt{sumXY}は例\ref{ex:new-main}のAの部分の\texttt{printf}のメッセージに合わせてある。次のようにして実行してみよう。
\begin{commandline2}
    \prompt \underline{./sumXY}\\
    Usage: sumXY InputFile OutputFile
\end{commandline2} \noindent
このとき\texttt{InputFile}, \texttt{OutputFile}の二つのコマンドライン引数が必要であるというエラーが表示される。例\ref{ex:new-main}のAの部分を変更することで、エラーメッセージは変更することができる。次のように実行しても同じメッセージが出力されるはずである。
\begin{commandline2}
    \prompt \underline{./sumXY input.dat}
\end{commandline2} \noindent
\begin{commandline2}
    \prompt \underline{./sumXY input.dat output.dat 10}
\end{commandline2} \noindent
一番目の例は引数が足りず、二番目の例は引数が多すぎるからである。例\ref{ex:new-main}では \texttt{argc} が3以外ならエラーメッセージを表示するようにしてあるが、
\begin{commandline2}
    \prompt \underline{./sumXY input.dat output.dat 10}
\end{commandline2} \noindent
は3個の引数だからエラーが出るのはおかしい、と考えるかもしれない。しかし、これは正しい動作である。なぜなら、引数としてプログラム名(\texttt{./sumXY})も1個と数えられるからである\footnote{\texttt{argv[0]}を用いると、例\ref{ex:new-main}のAのエラーメッセージは \texttt{printf("Usage: \%s InputFile OutputFile\textbackslash n", argv[0]);}と書くことができる。このようにしておくと、プログラム名をあらかじめソースコードに明示的に書き込む(「ハードコーディング」と呼ぶ)必要がなくなる。}。つまり、二番目の例では、\texttt{argc}の値は4であり、\texttt{argv}の中身は
\begin{quote}
    \begin{verbatim}
argv[0] = "./sumXY"の先頭アドレス
argv[1] = "input.dat"の先頭アドレス
argv[2] = "output.dat"の先頭アドレス
argv[3] = "10"の先頭アドレス
\end{verbatim}
\end{quote}
となっている。

\texttt{argc}が想定していた値と異なる場合、配列\texttt{argv}の有効範囲からはみ出した部分を参照して致命的なバグを発生させることがあるので、\texttt{argc}のチェック機構は面倒でもつけるようにしたほうがよい。

最後に、1行に2個の数字で10行程度書いた\texttt{input.dat}を作成し、次のように実行してみよう。\texttt{output.dat}に結果が書き込まれているはずである。
\begin{commandline2}
    \prompt \underline{./sumXY input.dat output.dat}
\end{commandline2}

\section{動的な配列の確保: \texttt{malloc}と\texttt{free}}
\label{sec:clang:malloc}

これまで、ポインタはあらかじめ確保した領域に対してのみ使ってきた。しかし、実行時に、入力値あるいは計算結果を反映して、50個のデータを取り扱いたいときもあれば、100個のデータを取り扱いたいときもある。もちろん、配列を利用してあらかじめ十分な大きさの領域(いまの場合は100個以上の数)を確保しておけばよいかもしれないが、平均的に10個ぐらいしか利用しないのに、たまに100個の場合があるからといって常に100個分の領域を確保しておくことは、無駄のように思える。これを回避するために、動的に、つまり、実行時に領域を確保する手段がある。これが、\texttt{malloc} (\textbf{m}emory \textbf{alloc}cationの略)である。

\subsection{\texttt{malloc}の使い方}

例\ref{ex:malloc}に\texttt{malloc}の使い方の雛型を示す\footnote{簡単のため、コマンドライン引数のチェックは行っていない。実際のプログラムでは、配列を確保する前に確保しようとしているサイズを確認すべきである。}。\texttt{malloc}は領域を確保できた場合はその領域の先頭アドレスを返すが、領域が確保できなかった場合は\texttt{NULL}を返す。\texttt{NULL}の場合は、何らかのエラー処理を行う必要がある。例\ref{ex:malloc}では、プログラムを強制的に終了している。\texttt{sizeof}関数は型の大きさ(バイト数)を調べる関数である。この場合は\texttt{int}の大きさ(通常4)を調べている。\texttt{double}を大きさを調べたい場合は\texttt{sizeof(double)}のように使う。例\ref{ex:malloc}では、\texttt{int} の領域を \texttt{n\_element} 個分確保したいので、\texttt{sizeof(int) * n\_element}を\texttt{malloc}に与えている\footnote{\texttt{malloc}はあらゆる型に対して使えるように要素数ではなくバイト数単位で必要メモリ量を指定する設計になっている。引数に\(0\)を与えてはならない。}。
\begin{reidai}\label{ex:malloc}
    \begin{verbatim}
#include <stdio.h>
#include <stdlib.h>

int main(int argc, char* argv[])
{
    const int n_element = atoi(argv[1]);
    int* array = (int*)malloc(sizeof(int) * n_element);
    if (array == NULL)
    {
        printf("Can't allocate memory.\n");
        exit(1);
    }
    for (int i = 0; i < n_element; ++i)
    {
        array[i] = i * i;
    }
    for (int i = 0; i < n_element; ++i)
    {
        printf("array[%d] = %d\n", i, array[i]);
    }
    return 0;
}
\end{verbatim}
\end{reidai} \noindent
なお、コンパイル時に、\texttt{malloc}の部分で``型の不整合'' という警告が出ることを防ぐために、\texttt{malloc}の返り値を明示的に\texttt{int*}型に変換している。これをキャスト(型変換)という。

\subsection{\texttt{free}の使い方}

例\ref{ex:malloc}には(引数の数のチェック以外にも)適切でない部分がある。\texttt{malloc}で確保した領域を解放していない点である。確保した領域を解放するためには、\texttt{free}という関数を使う\footnote{例\ref{ex:malloc}のようにプログラムがすぐに終了してしまう場合には、\texttt{free}は必要ないと主張する人がいるかもしれない。その主張も確かに間違っている訳ではないが、本当に必要なときに忘れないために普段から書く癖をつけるべきである。}。
具体的には、例\ref{ex:malloc}の \texttt{return 0;} の前に
\begin{verbatim}
     free(array);
\end{verbatim}
と1行書けばよい\footnote{\texttt{free}を同じ対象に対して複数回呼んではならない。}。例\ref{ex:malloc}を若干変更し、\texttt{free}が重要となる例を考えよう。なお、今回は\texttt{assert.h}を用いて簡易的な\texttt{argc}のチェック機構をつけてある(\texttt{argc == 2}が満たされないとエラーメッセージを表示したのち異常終了する)。
\texttt{assert}機能はデバッグの際極めて重宝するので、覚えておくとよい\footnote{もちろんチェックにはそれなりの計算コストがかかるが、\texttt{-DNDEBUG}をつけてコンパイルするとまとめて除去できるので安心してたくさん使ってほしい。}。
\begin{reidai}\label{ex:malloc-free}
    \begin{verbatim}
#include <assert.h>
#include <stdio.h>
#include <stdlib.h>

int main(int argc, char* argv[])
{
    assert(argc == 2);
    const int n_element = atoi(argv[1]);
    for (int j = 0; j < 3; ++j)
    {
        int* array = (int*)malloc(sizeof(int) * n_element);
        if (array == NULL)
        {
            printf("Can't allocate memory.\n");
            exit(1);
        }
        for (int i = 0; i < n_element; ++i)
        {
            switch (j)
            {
            case 0:
                array[i] = i;
                break;
            case 1:
                array[i] = i * i;
                break;
            default:
                array[i] = i * i * i;
            }
        }
        for (int i = 0; i < n_element; ++i)
        {
            printf("array[%d] = %d\n", i, array[i]);
        }
        free(array);
    }
    return 0;
}
\end{verbatim}
\end{reidai} \noindent
もし仮に\texttt{free} がない場合、\texttt{j = 1} のとき、\texttt{malloc} に成功すると \texttt{array} は新しく確保された領域の先頭アドレスを持つ。この時、\texttt{j = 0}で確保した領域はどうなったのだろうか? その領域は、このプログラム自身が確保した領域として、このプログラムが終了するまで解放されない。しかも、\texttt{array} はすでに \texttt{j = 0} のときに確保した領域を忘れているので、このプログラムからもその領域に適切にアクセスすることはできない。つまり、\texttt{j = 0} のときの領域はこのプログラムが持っているが、使うことができない領域として存在し続けることになる\footnote{これを「メモリリーク」と呼ぶ}。これは非常に無駄なことである。これを回避するためにも、使わなくなった領域は \texttt{free} することが重要である。もちろん、最近の計算機はメモリ(および、スワップ領域)をたくさん積んでいるので例\ref{ex:malloc-free}程度のプログラムなら\texttt{free}がなくても平気で最後まで動くはずであるが、習慣として、\texttt{malloc} した領域を使い終わったら必ず \texttt{free} するのがよい。

\subsection{動的な2次元配列}

C言語において、2次元配列は 1次元配列の先頭を指すポインタの配列として表すことができる。
すなわち、各行をそれぞれ1次元配列として考え、各行の先頭を指すポインタを別の1次元配列に格納しておけばよい。
この場合、後者の配列は要素がポインタ型の配列であるので、double型の2次元配列の場合には
\begin{quote}
    \begin{verbatim}
double** matrix;
\end{verbatim}
\end{quote}
と宣言する必要がある。その後、各行に対応する1次元配列を \texttt{malloc} する。
\begin{reidai}\label{ex:malloc-2dim}
    \begin{verbatim}
#include <stdio.h>
#include <stdlib.h>

int main(void)
{
    const int m = 3;
    const int n = 4;
    double** matrix = (double**)malloc(sizeof(double*) * m);
    if (matrix == NULL)
    {
        printf("Can't allocate memory.\n");
        exit(1);
    }
    for (int i = 0; i < m; ++i)
    {
        matrix[i] = (double*)malloc(sizeof(double) * n);
        if (matrix[i] == NULL)
        {
            printf("Can't allocate memory.\n");
            exit(1);
        }
    }
    for (int i = 0; i < m; ++i)
    {
        for (int j = 0; j < n; ++j)
        {
            matrix[i][j] = (i + 1) * (j + 1);
        }
    }
    for (int i = 0; i < m; ++i)
    {
        for (int j = 0; j < n; ++j)
        {
            printf("matrix[%d][%d] = %lf\n", i, j, matrix[i][j]);
        }
    }
    for (int i = 0; i < m; ++i)
    {
        free(matrix[i]);
    }
    free(matrix);
    return 0;
}
\end{verbatim}
\end{reidai} \noindent
この例では、\(3 \times 4\)の2次元配列(行列)を宣言している。まず、\texttt{matrix} に長さ3の1次元配列(要素はdouble*型)を確保する。さらに、各行を格納する長さ4の1次元配列(要素はdouble型)を3個確保し、その先頭アドレスを \texttt{matrix} に格納する。配列の\((i,j)\)成分へのアクセスは、静的な2次元配列の場合と同じく \texttt{matrix[i][j]} と書けばよい。\texttt{matrix[i][j]} は \texttt{*(*(matrix + i) + j)} と等価であることに注意せよ。\texttt{*(matrix + i)} により \(i\) 行目の先頭アドレスが得られ、それに \(j\) を足したアドレスの中身を参照することで、\((i,j)\)成分が得られる。確保したメモリの解放の際は、まず各行に対応する1次元配列を解放した後、最後にそれらの先頭先頭アドレスを格納していたdoubleポインタ型の1次元配列を解放する必要がある\footnote{解放の順番を間違うとエラーとなる。その理由を考えてみよ。}。

上記の方法で動的な2次元配列を取り扱えるようになるが、実際に行列としてLAPACKなどの数値計算ライブラリと組み合わせて使うには、二つの問題がある。一つ目は行列の要素の連続性の問題、二つ目は要素の並ぶ順番である。

LAPACKなどの数値計算ライブラリでは、行列の要素は全てメモリ上で連続であると仮定されている。しかし、上記の方法では、各行のデータは連続であるが、行と行の間が連続であるとは限らない。C言語では、連続した \texttt{malloc} の呼び出しで連続したメモリ領域が割り当てられることは保証されていないためである。メモリ上で \texttt{matrix[0][0]} の次には \texttt{matrix[0][1]}、\texttt{matrix[0][2]}、\texttt{matrix[0][3]} と並ぶが、その次が \texttt{matrix[1][0]} とは限らないのである。

この問題を解決するには、それぞれの行に対して \texttt{malloc} を行うのではなく、一度に\(3 \times 4 = 12\)の長さの1次元配列をまとめて確保した上で、4要素毎の要素(各行の先頭に対応)のポインタをポインタ型配列に格納していけばよい。具体的には、コードを以下のように修正する。
\begin{reidai}\label{ex:malloc-2dim-continuous}
    \begin{verbatim}
#include <stdio.h>
#include <stdlib.h>

int main(void)
{
    const int m = 3;
    const int n = 4;
    double** matrix = (double**)malloc(sizeof(double*) * m);
    if (matrix == NULL)
    {
        printf("Can't allocate memory.\n");
        exit(1);
    }
    matrix[0] = (double*)malloc(sizeof(double) * m * n);
    if (matrix[0] == NULL)
    {
        printf("Can't allocate memory.\n");
        exit(1);
    }
    for (int i = 1; i < m; ++i)
    {
        matrix[i] = matrix[i - 1] + n;
    }
    for (int i = 0; i < m; ++i)
    {
        for (int j = 0; j < n; ++j)
        {
            matrix[i][j] = (i + 1) * (j + 1);
        }
    }
    for (int i = 0; i < m; ++i)
    {
        for (int j = 0; j < n; ++j)
        {
            printf("matrix[%d][%d] = %lf\n", i, j, matrix[i][j]);
        }
    }
    free(matrix[0]);
    free(matrix);
    return 0;
}
\end{verbatim}
\end{reidai} \noindent
修正前のコードと異なり、\texttt{malloc} と \texttt{free} は、それぞれ2回ずつしか呼び出されていないことに注意せよ。

次に、二つ目の問題、要素の並ぶ順番について考える。LAPACKなどの数値計算ライブラリは、Fortran言語を用いて書かれていることが多い。Fortran言語では、2次元配列(行列)の要素は各列の要素がメモリ上で連続して並ぶ。例えば、\(3 \times 3\)行列では、要素はメモリ上で、\((0,0) \rightarrow (1,0) \rightarrow (2,0) \rightarrow (0,1) \rightarrow (1,1) \rightarrow (2,1) \rightarrow (0,2) \rightarrow (1,2) \rightarrow (2,2)\) の順に配置される。これを「列優勢 (column-major)」と呼ぶ。一方、C言語では、メモリ上で \texttt{A[0][0]} の次に配置されるのは \texttt{A[0][1]} である。こちらは「行優勢 (row-major)」と呼ぶ。
\begin{figure}[H]
    \centering
    \resizebox{0.60\textwidth}{!}{\includegraphics{major.pdf}}
\end{figure}
C言語上で行列の\((i,j)\)成分を \texttt{A[i][j]} に代入したもの(row-major)をLAPACKライブラリに渡すと、LAPACK側では要素がcolumn-majorで並んでいると解釈して計算を行う。すなわち、入力行列が転置されてしまう。また、計算結果もC言語から見ると転置された状態で返されることになってしまう。

これを防ぐには、行列の\((i,j)\)成分を \texttt{A[i][j]} ではなく \texttt{A[j][i]} に格納するように決めてプログラムを書けばよいのだが、「常に順番を逆に書く」というのはプログラム作成時に混乱してしまう恐れがある。一つの解決方法は、Cプリプロセッサのマクロ機能を使い、ソースコードの先頭で
\begin{quote}
    \begin{verbatim}
#define mat_elem(mat, i, j) (mat)[(j)][(i)]
\end{verbatim}
\end{quote}
のように \texttt{mat\_elem} 関数を定義することである\footnote{一般的にCプロセッサのマクロの多用は勧められていないが、背に腹は代えられない。}。これにより、プログラム中で行列の\((i,j)\)成分を \texttt{mat\_elem(A, i, j)} と書けるようになる。
\begin{reidai}\label{ex:malloc-2dim-column-major}
    \begin{verbatim}
#include <stdio.h>
#include <stdlib.h>

#define mat_elem(mat, i, j) (mat)[(j)][(i)]

int main(void)
{
    const int m = 3;
    const int n = 4;
    double** matrix = (double**)malloc(sizeof(double*) * n);
    if (matrix == NULL)
    {
        printf("Can't allocate memory.\n");
        exit(1);
    }
    matrix[0] = (double*)malloc(sizeof(double) * m * n);
    if (matrix[0] == NULL)
    {
        printf("Can't allocate memory.\n");
        exit(1);
    }
    for (int i = 1; i < n; ++i)
    {
        matrix[i] = matrix[i - 1] + m;
    }
    for (int i = 0; i < m; ++i)
    {
        for (int j = 0; j < n; ++j)
        {
            mat_elem(matrix, i, j) = (i + 1) * (j + 1);
        }
    }
    for (int i = 0; i < m; ++i)
    {
        for (int j = 0; j < n; ++j)
        {
            printf("matrix[%d][%d] = %lf\n", i, j, mat_elem(matrix, i, j));
        }
    }
    free(matrix[0]);
    free(matrix);
    return 0;
}
\end{verbatim}
\end{reidai} \noindent
プログラムの前半で、\(3 \times 4\)ではなく\(4 \times 3\)の2次元配列として行列を作成していることに注意せよ\footnote{列優勢(column-major)の2次元配列の確保や解放のための関数、および要素へのアクセス用のマクロ一式をまとめたものが、\hypertarget{cmatrix}{\texttt{cmatrix.h}} として\url{https://github.com/utphys-comp/cmatrix/} で公開されている。}。

\section{\texttt{segmentation fault}が出たら}
コンパイルは出来たのに、実行してみたら\texttt{segmentation fault}と出てプログラムが強制終了することがある\footnote{``することがある''というのが重要。プログラム自体が全く同じでも異常終了したりしなかったりする。}。こういった場合の典型的な原因と対処法を説明する。

\subsection*{スタックの枯渇}
\ref{ex:array1d}で説明したとおり、動的でない配列を大量に確保するとクラッシュすることがある。他にも関数の呼び出し過多(主に再帰)でも同様の現象が起こることがある。

\clangpara{対処法}
動的配列に変更する。再帰を\texttt{for}などのより簡単な処理に置き換える。

\subsection*{範囲外参照}
動的かそうでないかに関わらず、小さすぎる/大きすぎる値を\texttt{[]}に入れて配列を参照するとクラッシュすることがある。\texttt{malloc}の容量設定ミス(要素数ではなくバイト数)や\texttt{argc}の確認不備でよく発生する。

\clangpara{対処法}
ソースコードを見直す。(例えば1次元配列\texttt{a[N]}に対して参照可能な領域は\texttt{a[0], a[1]..., a[N-1]}だけである。)

\subsection*{ダングリングポインタの参照}
代入を行っていないポインタや\texttt{NULL}ポインタ、すでに解放済みの動的配列にアクセスするとクラッシュすることがある。\texttt{fopen}や\texttt{malloc}でよく発生する。

\clangpara{対処法}
ポインタにアクセスする前に、そのポインタがきちんと有効な対象を指しているかを確認する。\texttt{NULL}チェックを行う。

\section{外部ヘッダの利用}
ここでは、他の誰かが作成してくれたものを如何にして自分のプログラムに組み込むか、ということについて説明する。やや難しい内容であるので、読み飛ばしても構わない。

\subsection{外部ヘッダの利用}
既に何度か出てきているが、\texttt{.h}で終わるファイルをヘッダという。
ヘッダの役割の一つとして関数の宣言や定義\footnote{定義を書く場合にはやや特殊な工夫が必要になるので、自作する場合には気をつけよ。}を書くというものがある。
今回は先程出てきた\hyperlink{cmatrix}{\texttt{cmatrix.h}}を例として、外部ヘッダに書かれた関数の使用方法を説明する。

\url{https://github.com/utphys-comp/cmatrix/}から\texttt{cmatrix.h}をダウンロードしてきて、以下の例\ref{ex:ext_header}と\textbf{同じディレクトリ}に配置しよう。

\begin{reidai}\label{ex:ext_header}
    \begin{verbatim}
#include "cmatrix.h"
#include <stdio.h>

int main(void)
{
    const int N = 3;
    double** a = alloc_dmatrix(N, N);
    for (int j = 0; j < N; ++j)
    {
        for (int i = 0; i < N; ++i)
        {
            mat_elem(a, i, j) = i + j * N;
        }
    }
    fprint_dmatrix(stdout, N, N, a);
    free_dmatrix(a);
    return 0;
}
\end{verbatim}
\end{reidai}
ここで新しく出てきた\texttt{\#include "file"}の記法はただ通常の\texttt{\#include <file>}とヘッダ検索の順序が異なるだけのものである。

コンパイルにはいつもどおりでよい。

\begin{commandline2}
    \prompt \underline{gcc \ref{ex:ext_header}c -Wall -Wextra}
\end{commandline2} \noindent

\clangpara{コンパイラにヘッダの場所を教える}
先程は\texttt{.c}と同じ場所に\texttt{.h}を置いたので、特に何もせずともコンパイルができた。しかし実際にはもっと込み入った場所に置く必要が出てくる場合もある。
そんなときには以下の方法を試してみよう。
\begin{description}
    \item[明示的に場所を指定する]\mbox{}\\
          \texttt{gcc}に\texttt{-I}オプションを与えると、指定したディレクトリをヘッダのある場所として検索しに行くようになる。
    \item[\texttt{CPATH}を設定する]\mbox{}\\
          詳しく説明はしないが、\texttt{export}コマンドを使って\texttt{CPATH}という環境変数を改変することで\texttt{gcc}がヘッダを探索しに行く場所を増やすことができる。変更はターミナルからログアウトするまでずっと有効である。
    \item[シンボリックリンクを貼る]\mbox{}\\
          \texttt{ln -s}を使ってシンボリックリンクを作ると、全く別の場所にあるヘッダがあたかも\texttt{.c}の隣に置いてあるかのように偽装することができる。参照先のファイルを変更すると手元にあるリンクも自動でそれに追従するので、コピーするよりもスマートである。
\end{description}

\subsection{外部ライブラリの利用}
次は同じようなことをライブラリに対してやってみよう。例として何度か出てきているLAPACKを使用する。標準的な環境では特に何もせずとも使えるはずであるが、以下のサンプルが動かないようなら自分の環境に合わせた方法でインストールすること。

以下の例\ref{ex:ext_lib}を\texttt{cmatrix.h}と同じディレクトリに配置しよう。

\begin{reidai}\label{ex:ext_lib}
    \begin{verbatim}
#include "cmatrix.h"
#include <assert.h>
#include <stdio.h>

extern void dgesv_(const int*, const int*, double*, const int*,
    int*, double*, const int*, int*);

int main(void)
{
    const int N = 3;
    const int NRHS = 1;
    const int LDA = N;
    const int LDB = N;
    double** a = alloc_dmatrix(N, N);
    double* b = alloc_dvector(N);
    int* ipiv = alloc_ivector(N);
    for (int i = 0; i < N; ++i)
    {
        mat_elem(a, i, i) = 1;
        if (i + 1 != N)
        {
            mat_elem(a, i, i + 1) = -1;
        }
    }
    b[N - 1] = 1;
    int info;
    dgesv_(&N, &NRHS, mat_ptr(a), &LDA, vec_ptr(ipiv),
        vec_ptr(b), &LDB, &info);
    assert(info == 0);
    fprint_dvector(stdout, N, b);
    free_dmatrix(a);
    free_dvector(b);
    free_ivector(ipiv);
    return 0;
}
\end{verbatim}
\end{reidai}
今回は\texttt{-llapack -lblas}のオプションが必要になる。LAPACKがBLASに依存している都合上、\texttt{-l}の順番を変えてはならない。

\begin{commandline2}
    \prompt \underline{gcc \ref{ex:ext_lib}c -llapack -lblas -Wall -Wextra}
\end{commandline2} \noindent

ここで何をしているのかについては説明しないが、\url{https://netlib.org/lapack/explore-html/index.html}を使えば調べられる。
最初に書いてある\texttt{dgesv\_}の宣言部分は\ref{ex:fort}でやっていることと本質的に同じであるが、LAPACKのソースを見て引数の型を調べるのは大変なので、そのときにもこのページが役に立つだろう\footnote{安全性のためFortranにおける入力専用変数にはすべて\texttt{const}をつけてみたが、難しかったら取り除いても構わない。}。

\clangpara{コンパイラにライブラリの場所を教える}
今回何も工夫をせずともコンパイルができたのは、LAPACKがシステムの奥深くに配置されており、その場所が\texttt{gcc}の自動探索対象箇所であったからである\footnote{インストールの方法によってはそういった場所に置いてくれないこともある。}。よってこれほどすんなりとは行かないことも十分ありえる。
そんなときには以下の方法を試してみよう。
\begin{description}
    \item[明示的に場所を指定する]\mbox{}\\
          \texttt{gcc}に\texttt{-L}オプションを与えると、指定したディレクトリをライブラリのある場所として探索しに行くようになる。
    \item[\texttt{LD\_LIBRARY\_PATH}を設定する]\mbox{}\\
          詳しく説明はしないが、\texttt{export}コマンドを使って\texttt{LD\_LIBRARY\_PATH}という環境変数を改変することで\texttt{gcc}がライブラリを探索しに行く場所を増やすことができる。変更はターミナルからログアウトするまでずっと有効である。
\end{description}

\section{外部ツールの利用}
C言語でのプログラミングにおいて有用な外部ツールをいくつか紹介する。あくまで``紹介''に留めるので、使い方は各自調べること。
\begin{description}
    \item[make]\mbox{}\\
          コンパイル用コマンドを管理するツール。
    \item[cmake]\mbox{}\\
          コンパイルの設定、依存関係、テストなどを管理するツール。
    \item[git]\mbox{}\\
          変更履歴を管理するツール。
    \item[gdb]\mbox{}\\
          デバッガ。指定した行で処理を止めたり、変数の中身を覗いたり書き換えたりできる。
    \item[valgrind]\mbox{}\\
          メモリリークや範囲外参照を検出できるツール。
    \item[プロファイラ]\mbox{}\\
          プログラムの処理に時間を要している部分を解析するツール。gprofやvtuneなどが有名。
    \item[フォーマッタ]\mbox{}\\
          インデントやスペース等を全自動で揃えてくれるツール。C用としてはclang-formatが有名。
\end{description}

\section{練習問題の解答例}

以下に練習問題の解答例を示す。さまざまなプログラムの書き方があるので、これらの解答例にこだわらないこと。全く分からない場合に参考にする程度が望ましい。
\begin{renshuu-answer}{prob:2-1}
\baselineskip=12pt
\begin{verbatim}
#include <stdio.h>

int main(void)
{
    /* ax+b=0 */
    const double a = 4.0;
    const double b = 1.0;
    printf("ax+b=0\n");
    printf("  a = %lf\n", a);
    printf("  b = %lf\n\n", b);
    if (a == 0.0)
    {
        if (b == 0.0)
        {
            printf("  x = all\n");
        }
        else
        {
            printf("  x = nothing\n");
        }
    }
    else
    {
        printf("  x = %lf\n", -b / a);
    }
    return 0;
}
\end{verbatim}
\end{renshuu-answer}
\begin{renshuu-answer}{prob:2-2}
\baselineskip=12pt
\begin{verbatim}
#include <stdio.h>
#include <stdlib.h>

int main(void)
{
    /* calculate "n!" */
    const int n = 10;
    if (n < 0)
    {
        printf("Can't calculate n!.\n");
        printf("n = %d\n", n);
        exit(1);
    }
    int ans = 1;
    for (int i = 1; i <= n; ++i)
    {
        ans *= i;
    }
    printf("%d! = %d\n", n, ans);
    return 0;
}
\end{verbatim}
\end{renshuu-answer}
\begin{renshuu-answer}{prob:2-3}
\baselineskip=12pt
\begin{verbatim}
#include <stdio.h>

int main(void)
{
    /* n madeno sosuu. */
    const int n = 100;
    printf("%d madeno sosuu = ", n);
    int ans = 2;
    while (ans <= n)
    {
        int flag = 1;
        for (int i = 2; i < ans; ++i)
        {
            if (ans % i == 0)
            {
                flag = 0;
                break;
            }
        }
        if (flag == 1)
        {
            printf("%d, ", ans);
        }
        ++ans;
    }
    printf("\n");
    return 0;
}
\end{verbatim}
\end{renshuu-answer}
\begin{renshuu-answer}{prob:3-1}
\baselineskip=12pt
\begin{verbatim}
#include <stdio.h>

int main(void)
{
    int a[5] = { 1, 2, 3, 4, 5 };
    for (int i = 0; i < 5; ++i)
    {
        printf("Start: a[%d] = %d\n", i, a[i]);
    }
    int b[5];
    for (int i = 0; i < 5; ++i)
    {
        b[i] = a[i];
        a[i] = 0;
        printf("End  : a[%d] = %d, b[%d] = %d\n", i, a[i], i, b[i]);
    }
    return 0;
}
\end{verbatim}
\end{renshuu-answer}
\begin{renshuu-answer}{prob:3-2}
\baselineskip=12pt
\begin{verbatim}
#include <stdio.h>

void seki(double a[2][2], double b[2][2], double c[2][2])
{
    c[0][0] = a[0][0] * b[0][0] + a[0][1] * b[1][0];
    c[0][1] = a[0][0] * b[0][1] + a[0][1] * b[1][1];
    c[1][0] = a[1][0] * b[0][0] + a[1][1] * b[1][0];
    c[1][1] = a[1][0] * b[0][1] + a[1][1] * b[1][1];
}

int main(void)
{
    double a[2][2]
        = { { 1.0, 2.0 },
              { 3.0, 4.0 } };
    double b[2][2]
        = { { -1.0, -2.0 },
              { -3.0, -4.0 } };
    for (int i = 0; i < 2; ++i)
    {
        for (int j = 0; j < 2; ++j)
        {
            printf("a[%d][%d] = %lf, b[%d][%d] = %lf\n",
                i, j, a[i][j], i, j, b[i][j]);
        }
    }
    double c[2][2];
    seki(a, b, c);
    for (int i = 0; i < 2; ++i)
    {
        for (int j = 0; j < 2; ++j)
        {
            printf("(a x b)[%d][%d] = %lf\n", i, j, c[i][j]);
        }
    }
    return 0;
}
\end{verbatim}
\end{renshuu-answer}
\begin{renshuu-answer}{prob:4-1}
\baselineskip=12pt
\begin{verbatim}
#include <stdio.h>
#include <stdlib.h>
#include <string.h>

int main(void)
{
    /* calculate "n!" */
    int n;
    printf("Input(int): ");
    scanf("%d", &n);
    if (n < 0)
    {
        printf("Can't calculate n!.\n");
        printf("n = %d\n", n);
        exit(1);
    }
    int ans = 1;
    for (int i = 1; i <= n; ++i)
    {
        ans *= i;
    }
    printf("%d! = %d\n", n, ans);
    return 0;
}
\end{verbatim}
\end{renshuu-answer}
\begin{renshuu-answer}{prob:5-1}
\baselineskip=12pt
\begin{verbatim}
#include <stdio.h>

int main(void)
{
    int array[10];
    int* p;
    for (int i = 0; i < 10; ++i)
    {
        array[i] = i * i;
    }
    for (int i = 0; i < 10; ++i)
    {
        printf("array[%d] = %d\n", i, *(array + i));
    }
    for (int i = 0; i < 10; ++i)
    {
        p = &(array[i]);
        printf("array[%d] = %d\n", i, *p);
    }
    return 0;
}
\end{verbatim}
\end{renshuu-answer}
\begin{renshuu-answer}{prob:7-1}
\baselineskip=12pt
\begin{verbatim}
#include <stdio.h>

struct date
{
    unsigned day, month;
    unsigned days;
};

int main(void)
{
    const unsigned nMonth[12] = { 31, 28, 31, 30, 31, 30,
        31, 31, 30, 31, 30, 31 };
    struct date inputDay;
    printf("Input(month & day) :\n");
    printf("             month :");
    scanf("%d", &(inputDay.month));
    printf("             day   :");
    scanf("%d", &(inputDay.day));
    inputDay.days = 0;
    for (int i = 1; i < inputDay.month; ++i)
    {
        inputDay.days += nMonth[i - 1];
    }
    inputDay.days += inputDay.day;
    printf("%d/%d = %d days from 1/1\n", inputDay.month,
        inputDay.day, inputDay.days);
    return 0;
}
\end{verbatim}
\end{renshuu-answer}
